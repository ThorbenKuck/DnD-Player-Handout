\documentclass[10pt,twoside,twocolumn,openany]{book}
\usepackage[bg-letter]{dnd} % Options: bg-a4, bg-letter, bg-full, bg-print, bg-none.
\usepackage[english]{babel}
\usepackage[utf8]{inputenc}

\setcounter{secnumdepth}{1}
\setcounter{tocdepth}{1}

\makeindex
\newcommand{\darkvision}[1]{You can see within #1 feet of dim light as if it where bright light, and in darkness as if it where dim light. You can't  discern color in darkness, only shadows of grey.}
% Start document
\begin{document}
\fontfamily{ppl}\selectfont % Set text font

\topskip0pt
\vspace*{\fill}
\begin{center}
{\Huge \textbf{Player-Handout}}
\end{center}
\vspace*{\fill}
\newpage

% Your content goes here

\tableofcontents
\newpage

\chapter{Chapter1: The World}
\section{Gobink}

\subsection{Sobra}
Sobra ist die Heimat der Men-
schen und Weylyns (eine art evolutionäre
zwischenstufe zwischen den Menschen und
den Werwölfen). Während die Weylyns al-
lerdings eher instiktgetriebene Wesen sind
und in dem Forrwar leben, einem riesiegen,
verwilderten Wald, haben die Menschen ei-
ne gigantische Stadt namen Golham ge-
baut. Von einer Robusten Mauer umge-
ben, ist sie ein sicherer Hafen für Wande-
rer und Abentuerer, aber auch für Händ-
ler und Schmieden. Natürlich gibt es aber
auch noch andere Dörfer über Sobra ver-
teielt. In der Mitte zwischen dem Reich der
Menschen und dem der Weylyns liegt die
Tundra Sobras, ein viel umkämpfter Platz.
Trostlos und voller überreste und Errinne-
rungen der Vergangenen kämpfe. Am nörd-
lichsten Punkt von Sobra liegt Aeneas, ein
Handelshafen und Kontakt-Punkt zu ande-
ren Kontinenten.

\subsection{Gilbrit}
Gilbrit beheimatet einen Berg na-
mens "Limbür mar", ein Berg, etwa ge-
formt wie ein Tropfen, oder ein Haken. Er
ist die Heimat der Gnome. Gnome sind ein
sehr in sich verschlossenes Folg. Es ist allge-
mein wenig über sie bekannt und das weni-
ge was bekannt ist, lernten Wanderer, Bar-
den und Abenteurer von Elfen, welche auf
Gilbrit wohnten oder Studien durchführ-
ten.

\subsection{Valtras}
Valtras ist ein sehr Kalter Kon-
tinent. Er liegt am Südlichsten Punkt von
Goobink. Auf ihm trohnt ein Vulkan na-
mens ärau buhr mar". Er brodelt schon seid
jahrhunderten, brach aber niemals aus. Le-
genden ranken sich um diesen Vulkan, aller-
dings macht keine dieser Legenden wirklich
Sinn. Direkt an arau buhr mar angrenzend
liegt die Krakamur-Bergkette. In ihr befinden
sich viele Schmieden und Gewölbe, welche
als die Heimat für Zwerge fungieren.

\newpage

\subsection{Nobrut}
Nobrut hingegen ist die Heimat
der Gombruts (eine, von den Menschen ab-
stammende und den Riesen ähnelnde Ras-
se. Sie werden so alt, dass sie teils von Stein
überzogen sind und gelten allgemein als
Harmlos aber Tollpatschig) und Elfen. Ele-
nera ist ein Baum, welcher als Heimat der
Elfen gilt und im Osten Nobruts liegt. In
dem, Elenear umgebenen, Wald leben viele
unterschiedliche Arten von Elfen, verstreut
über den Wald. Anderen Rassen wissen
nicht viel über den Wald, lediglich erfahre-
ne Wanderer und Händler anderer Rassen
kennen den Weg zu Elenear und anderen
Dörfern in dem Wald. Goraba ist eine Berg-
Kette, ungefähr der Große von dem Forr-
war und Westlich auf Nobrut gelegen. Sie
ist die Heimat der Gombruts. Seid einiger
Zeit versuchen auch Zwerge dort eine Hei-
mat zu änden, auf der Suche nach Unter-
kunft, oder simpel auf der Suche nach sel-
tenen Materialien. Zu mindestens hört man
dies, wenn ein Barde sich bei einem Krug
Meet erboßt, euch eine dieser Geschichten
zu erzählen.

\newpage
\section{Dragonborns}
\paragraph{Origin}

Nobrut wurde in der Vergangenheit von Drachen und Drachen-Geborenen beheimatet. Die Hauptstadt der Drachengeborenen hieß Drakor. Auch wenn Drakor heute nur noch aus einige Ruinen besteht, welche aus dem Sand Gombruts ragen, so war es einst eine Pompöse Stadt, an der alle Wesen willkommen waren. 

Als erstes verschwanden die Drachen. Die Drachengeborenen lebten friedlich in der Mitte von Nobrut und bekamen dieses nicht mit. Doch nach nur ein paar Jahren wurden auch diese Überrascht. Zu erst wurden die kleinen Dörfer, nahe der Goraba-Bergkette überrant. Die Gombrut, immer auf der Suche nach Nahrung überranten zu erst die kleineren Dörfer. Ein Krieg brach aus und das Obwohl die Drachengeborenen ein friedliches Volk waren. Die Gombrut hatten Angst vor den Drachengeborenen, da sie Nachfahren der mächtigen Drachen waren, doch die Wahrheit war, dass die Drachengeborenen ohne die mächtigen Drachen verloren waren. Die Drachen tauchten jedoch nicht wieder auf und die Drachengeborenen waren dem Untergang geweit. Selbst Versuche, ein Abkommen zwischen den Gombrut und den Drachengeborenen zu verabschieden scheiterten. Drakor wurde über einen Zeitraum von 9 Monaten belagert und beide Seite verloren viele Krieger und Kampf-Magier. Nach den 9 Monaten hielten die Drachengeborenen immer noch ihre Stellung, aber dennoch erschien die einzige Möglichkeit, wie die Drachengeborenen überleben konnten, der Rückzug zu sein. Die Mauern Drakor's fielen und der Herrscher Ayyaam, ging zusammen mit ihnen unter, während er zusammen mit einigen tapferen Soldaten, die Gombruts lange genug zurück hielt, damit sein Folk fliehen konnte.

Heute leben die Drachengeborenen verstreut über ganz Goobink, mal in kleineren Dörfern, mal in mitten von anderen Wesen. Eines teilen sie allerdings bis zu dem heutigen Tage: Einen Hass auf die Gombruts und ein Streben in ihrem Herzen, Drakor wieder zu erbauen und den Drachengeborenen wieder ihren rechtmäßigen Platz auf Norbut zurück zu geben.
\newpage
\section{Nightelves}
\paragraph{Origins}

Die \textbf{Dark Elfen} finden sich in vielen Orten, großteils über Sobra. Ursprünglich fanden sie sich um Elenear, der Heimat der Hoch Elfen, doch sie wurden verstoßen nachdem einige der Hoch Elfen auszogen, um in den Wäldern um Elenear zu leben.

Ausgestoßen und voller Wut zogen sie in den Westen Nobruts, nur um sich mitten in einen Krieg zwischen den Drachengeborenen und den Gombruts wieder zu finden. Sie flohen zusammen mit den Drachen geborenen auf die Kontinente.

Einige dieser \textbf{Dark Elfen} landeten so auf Sobra. Anders als die Drachengeborenen allerdings, zogen diese nicht in die Städte und versuchten dort eine neue Heimat zu finden, sondern sie gingen in den Westen Sobras, durch die Tundra Sobras bis in den Forrwar. Als sie einen Höleneingang zu einem großen, unterirdischen Hölensystem fanden, entschlossen sie sich, dort nieder zu lassen.

Einige Jahre später betraten die \textbf{Dark Elfen} das erste mal wieder die Oberfläche. Sie hatten lange Zeit von allem Gelebt, was sich in der Höle finden ließ, doch es war Zeit, andere Nahrung zu finden. Der Wald jedoch war zu gefährlich. Viele die dort hinein gingen gerieten in Mitten von Weylin-Kreise, welche ebenso auf der Suche nach Nahrung waren. Deswegen entschlossen sie sich, in der Tundra nach Nahrung zu suchen. Was sie fanden war allerdings nicht ganz was sie suchten.

Sie fanden vereinzelnte Normaden Völker. Immer auf der Reise und auf der Flucht vor dem weigen Krieg zwischen den Weylins und den Menschen, war es üblich, dass Normaden hier nicht länger als 2 Tage an einem Ort verblieben. Überrascht von den Artefakten und schätzen, die die \textbf{Dark Elfen} in dem Besitz der Normaden sahen, griffen sie an. Sie schlachteten eine Gruppe nach der anderen ab und stahlen deren Wertsachen.

Auf Basis dieser Wertsachen etablierten die \textbf{Dark Elfen} einige "Handels-Verbindungen" mit vorbeiziehenden Wesen, sei es nun ein Krieger, welcher sich an der Front mit Weylins herum schlug oder ein armer Priester, welcher am Strand von der Tundra nach einem Schiffsbruch erwachte.

\chapter{Chapter 2: Races}
\section{Half-Gombrut}
\textit{As Grar realized, that not everybody was thinking good about his heriatage, he ran. Even tho, he was  stronger than any human, he was nowhere accepted into any kind of  society. He fought in a 100 Battles and killed more than any human, but his heriatage was his doom. Not as smart as he was, he walked straigth into a trap, build by 2 mortal enemys: A Gombrut and a Human}

\subsection{Left Behind}
Half-Gombruts' look like Humans, in a gray-isch skin. They have strong, well defined muscels and are more tall than wide. They apear as if a Human was streched out to far. Most of them are around 7 feet tall, yet none of these is smaller  than $6^{1/2}$ feet. Yet, because of the weigh only around 175 - 215 pounds. Because they are related to the Gombrut, the oldes Half-Gombruts are around 1000 years old and mature at around 100 years. The older a Half-Gombrut becomes, the more it is overdrawn with Stone, as if the creature becomes one with the mountain.

They originaly derive from Humans, yet they are way less intelligent than humans. Some going so far, as being as stupid as a simple bug. Theire clumsyness is infamous. They almost always fail at basic task, such as grabing berrys from a small tree in a spectacular fasion. Most of the time, the only trouble they have, is the trouble they get themself into, if they are to clumsy to correctly performe any task.

Even though they are ousiders in nearly any society, they are verry kind and good creatures, healping anyone in need.

\subsection{Antisocial Genes}
Because neither the Humans nor the Gombruts are willing to accept a Half-Gombrut into theire society, Half-Gombruts are antisocial loners in theire nature. Some form small tribes and some integrate themself into smaller groups, but most of them are always on theire own at adventures or living in caves, naturaly acouring in mountains.

They are friendly towards all creatures, yet they know to defend themself. They inherited this trait from theire Gombrut ancestors. Half-Gombruts thrief towards building up theire own perfect world, just like theire Human ancestors. They hold true to most of the prejudice against Humans and against Gombruts. 

\newpage
\begin{commentbox}{Uncommon Races}
The Half-Gombrut and every other Race in this Chapter are uncommon. They don't exist in every world of D\&D and are less widespread than other races.\\
\textbf{Half-Gombrut.} Half-Gombruts are verry rare and scattered over nearly all continents. They are so rare because most of them are never leaving there home-mountain.\\
\textbf{Half-Weylins.} Half-Weylins tend to live in forrests and socialice as little as possible, which is why they are almost never seen in many cities.\\
\end{commentbox}

\subsection{Half-Gombrut Names}
Half-Gombruts earn theire name from theire parents, when they are born. This name consists 2 parts. A name, and a family name. More often than not, Half-Gombruts have two family names, because theire parents had two different. Once they mature, most Half-Gombruts tend to give themself a new name, while keeping theire family name(s).

\begin{paperbox}{The names of Half-Gombruts}
Half-Gombruts are proud of there names, often reffering to the name as "the most beautiful thing in the world", even tho it sounds more like a stone rolling at an uneven surface. As you choose you name, keep that in mind.
\end{paperbox}

\textbf{Male Names:}
Dombra, Grutro, Ernero, Meratro, Utragro, Pretegre, Kumanrare
\textbf{Female Names:}
Arear, Zurare, Undradar, Kirantar, Hubrate, Gultar, Lurare
\textbf{Family Names:}
Stoneeater, Rockseeker, Meatservant, Goldfinder, Crowholder, Mountainclaimer, Dirtgrower

\subsection{Half-Gombrut Traits}
Your heritage manifests in different forms.

\textbf{Ability Score Increase.} Your Constitution score increases by 2, and your Strength score increases by 2. Your Intelligence decreases by 1.

\textbf{Age.} Your Half-Gombrut ages slowly. They atain the development off a 10-year-old Human by the age of 60 and reach adulthot by 100. They live to be up to 1000 Years.

\textbf{Alignmennt.} Half-Gombruts tend towards a Good alignment, verry rarely if ever dipping into the Chaotic or even Evil alignment. Yet they are not bound to rules of any kind, except they put on themselfs. Most Half-Gombruts lean towards a Neutral or Chaotic alignment.

\textbf{Size.} Half-Gombruts are tall and slim creatures, even tho they are verry muscular. They are around 7 - $7^{1/4}$ feet tall, but never fall beneth $6^{1/2}$ feet.

\textbf{Speed.} Your base walking speed is 25 feet.

\textbf{Hardened Skin.} Because of your age and your diet, you have "stone in your blood". Your skin is verry thick and not as easy penetrable as most other skin. Your base AC is 12 instead of 10.

\textbf{Damage Resistance.} You have resistance to slashing damage.

\textbf{Fight Focus.} A Half-Gombrut can concentrate so hard, that the stone on his body changes the position. Your armor-class increases by 2 (after calculation). Once you used this trait, you can't use it again until you finish a long rest.

\textbf{Languages.} You can speak, read and write Common and Gombrut. Common though is a bit stuttering. You don't socialize much, so you can't practice this languages and with your parents you mostly speak Gombrut. So you are switching, leaving out words and changing the basic structure of some sentences.

\textbf{Darvision} Because you live in pitch black environments most of the time, you have superior vision in dim and dark conditions. \darkvision{60}

\textbf{Ancestory} You have proficiency in the History skill.

\begin{commentbox}{Playing an Half-Gombrut}
Half-Gombruts would act clumsy or awkward in most situations of social interactions without realising it. They never learned it, so they don't know it better. But mostly they don't care, so they never will learn it anyways.
\end{commentbox}

\newpage\mbox{}
\newpage
\section{Half-Weylin}

\textit{As Kraa reached the east point of the forwarr, he stumbled upon some remains of the fight, that had just ended. Kraa was sick of this all. He wanted to get away fom his old life. After nearly 4 Weeks in on the run, Kraa encountered a Human. Happy and full of hope he asked for a helping hand to get out of this nightmare, but only found the tip of a long sword, piercing his instesticals. "Why?", was the last word out of his mouth.}

\subsection{Wild Anatomy}
Half-Weylins are looking like a verry harrie hair, with a twist. They appear wolf-like in many forms and most of the time, they could  be mistaken for one. Other than Werewolfs tho, theire faces and Hands appear Human like. The hip of an Half-Weylin is formed in a way, that the Half-Weylin can walk on 2 feet just as good as on his feet and hand combined, like a wolf.

They come in at around $4^{3/4}$ - $5^{1/2}$ feet, a bit smaler than the average Human and a bit bigger than the average Weylin. They have a weird inner antonomy, which is not as reliably and consistantly working as a Humans. Most of them grow up to become 60 years old. Because they live in the woods and have to grow up quickly, they mature at around 1 - 2 years and start hunting prey. At around 8 years, nearly all Half-Weylins have learned to walk only on there 2 feet, but they take around 2 more years to get used to it.

Tho there heritage is both Human and Weylin, they tend to deny there Weylin heritage and adopting Human-history. Even there inteligence contests that, of the average Human. Thats why most of them lean twoards inflicting non-physical damage. But do not underistimate the strength, they can bring up, if needed. Even tho they akt within the boundries of what is considered to be "good", they are not likedd by humans because of there Weylin heritage.

\subsection{Pack-Live}
Half-Weylins which live in the east of sobra or on a nother continent all together tend to live the life of a loner, feeding on prey that they encounter on a dayli bases. But those, who have the luck to live with there own, do so. They form small Packs, which consists 1 alpha and every other Half-Weylin beneth him. Tho he is the alpha, Half-Weylins tend to have some sort of democracy in order.

Some of them live in houses, whilest others tend to sleep outside in the wilderness. This is nothing more than a taste realy, depending on the individual Half-Weylin. Most of these Half-Weylins believe nothing in terms of gods, but if they follow a god, they do it with great passion.

\begin{paperbox}{The Curse of Ancestory}
Most Humans dont like Weylins, but some hate any form of Weylin to the depths of there heart. Expect to get some rude comments along the way of the adventure or, in the worst case, even random attacks from Humans.\\
That includes: 
\begin{itemize}
\item Humans, that served in an army fighting against the Weylins.
\item Humans, that lost somebody important to them in a fight against Weylins.
\item Other creatures, that where affected by the war between Humans and Weylins.
\end{itemize}
Your character knows that and would most likely acts humble in such situations, to strengthen the good reputation of Half-Weylins. Except if your Character does not care about that or is marked by the daily hate against him.
\end{paperbox}
\subsection{Half-Weylin Names}
A typical Half-Weylin names constist of a name, given at birth and the pack he grew up in.\\\\
\textbf{Male Names:}\\
TODO\\
\textbf{Female Name:}\\
TODO\\
\textbf{Pack Names:}\\
TODO\\

\subsection{Half-Weylin Traits}
Your heritage manifests in different forms.

\textbf{Ability Score Increase.} Your Wisdom score increases by 2 and your Intelligence score increases by 1.

\textbf{Age.} Half-Weylins age verry fast. They live to be 60 years old and maturing at around 1 year.

\textbf{Alignmennt.} Half-Weylins tend to start with a neutral alignment, changing to evil as they age and observe the hostile and evil world around them. Half-Weylins living in a pack adopt the lawfull alignmennt and loners mostly adopt the chaotic alignment.

\textbf{Size.} Half-Weylins anatomy resemble partly that of a wolf. There size lies between $4^{3/4}$ - $5^{1/2}$, but mostly in the middle of this range.

\textbf{Speed.} Your base walking speed on foot is 20 feet and 40 on all fours.

\textbf{Versatile Walker.} As an action, you can choose to change the way you walk to \textit{"on two foot"} or \textit{"on all fours"}. However, you can't hold anything in your hands while on all fours. Wearing armor is okay, as long as it does not hinder the walking on all fours.

\textbf{Keen Senses.} You have proficiency in Perception check. In addition, you have advantage on perception checks that rely on smell or hearing.

\textbf{Claws.} Your claws are natural weapons, which you can use to make unarmed strikes. If you hit with them, you deal slashing damage equal to 1d4 + your Strength modifier, instead of bludgoning damage normal for an unarmed strike.

\textbf{Primal Instinkts} A Half-Weylin, thats currents hitpoints is beneth half his maximum hitpoints (rounded up) can use this trait to take an reaktion, whenever he takes damage. He can choose to walk 5 feet, in the direction of the source of the damage. After you heald above half the maximum hitpoints (rounded up), you have to take a long rest, bevore you can use this feature again.

\textbf{Darvision.} Because you live in a forrest, you have better sight in the dark than normal. \darkvision{30}

\textbf{Languages.} You can speak, read and write Common, Weylin and one language of your choice.

\begin{commentbox}{Playing an Half-Weylin}
Half-Weylins know about natures law and tend to leave other creatures as they are. Conflicts are nothing realy new to them, but they mostly keep out of the way of trouble relating to there Weylin ancestory. If your character is harrased, because he has a Weylin heritage, he would most likly not engage further into this debate.
\end{commentbox}

\chapter{Chapter 3: Classes}
\twocolumn[{%
\begin{@twocolumnfalse}
\header{The Necromancer Table}
\begin{dndtable}[XXX]
\textbf{Level} & \textbf{Proficiency Bonus} & \textbf{Feature} \\
1st & +2 & Controll Undead, Raise Undead\\
2st & +2 & Lifetap\\
3st & +2 & Archtype\\
4st & +2 & Ability Score Improvement\\
5st & +3 & Charnel Touch, Aura of Undead\\
6st & +3 & Archtype feature\\
7st & +3 & Dark Energy Burst\\
8st & +3 & Ability Score Improvement\\
9st & +4 & Control and Raise Undead Improvement\\
10st & +4 & Archetype Feature\\
11st & +4 & Aura of Undeath Improvement\\
12st & +4 & Ability Score Improvement\\
13st & +5 & Dominate Undead, Sacrifice Undead\\
14st & +5 & Archtype feature\\
15st & +5 & Ethereal Mind\\
16st & +5 & Ability Score Improvement\\
17st & +6 & Aura of Undeath Improvement\\
18st & +6 & Raise Undead Improvement\\
19st & +6 & Ability Score Improvement\\
20st & +6 & Death Gerneral
\end{dndtable}
\end{@twocolumnfalse}
}]

\section{Necromancer}
While others use magic to do paltry things like conjure fire or fly, the Necromancer is a master over death itself. They study the deep and forbidden secrets that raise the dead, controlling minions toward a variety of goals. Perhaps they seek the power that mastery over death provides. Perhaps they are serious and unashamed scholars, who reject the small-minded boundaries held to by others. Each enemy they fell becomes an eager and disposable ally, they become immune to the energies of death and decay, and ultimately harness the immortality and power of undeath for themselves.

\subsection{Undead Minions}
Necromancers exert control over undead creatures as if they were part of the same creature. Unlike hordes of wild undead, the undead controlled by a necromancer act as one unit, often running (or shambling) to each other’s aid when a member of the horde is injured or destroyed.


\subsection{A Life of Consequences}
Necromancers don’t have the best history with people. Though the large majority of necromancers merely want to help people, whether it be in communicating with dead family members or overthrowing tyrannical kings, evil necromancers are by far more well known for the occasional invasion of a peaceful city, which people don’t tend to take kindly to.
\newpage
A necromancer must always be careful of his company, because while many people will seek to understand his intentions, just as many will ignore them altogether in the assumption he is evil.

\subsection{Creating A Necromancer}
A Necromancer is a caster that is able to expel negative energies flowing through their veins. Necromancers are similar to sorcerers, but are more adept with necromancy and, to some extent, enchantment spells. They use their abilities to gain absolute control over their enemies' bodies, minds and souls. Often the best way to do this is by raising/summoning undead from their fallen enemies; a skill at which they are unparalleled. Necromancers are also effective with diseases, poison spells, and affecting opponents with fear, fatigue, exhaustion, pain, negative energy damage, or even gaining mindless supporters through the use of enchantment spells like charm or dominate.

A Necromancer's strengths are in bolstering undead, summoning or raising undead minions (which they can control up to a number of a large mob) and being able to cast a vast repertoire of various necromancy spells. They are strong spell casters but are not durable in physical combat. A Necromancer should primarily be used for crowd control, able to curse the enemy while animating different undead to occupy the enemy while their teammates continue to sustain a mass of dead bodies for you.

The most important thing to consider when creating a necromancer are the reasons you became a necromancer. Were you driven to the edge when someone you love was killed or did an oppressive government lead to your seeking rebellion? Either way, a necromancer is often misunderstood by those around them. Necromancers must usually hide their abilities to avoid persecution.

\subsubsection{Quick Build}
You can make a warlock quickly by following these quggestions. First, Charisma should be your highest ability score, followed by Constitution. Second, choose the Sage background.

\subsubsection{Class Features}
As a Necromancer, you gain the following class features.

\paragraph{HIT POINTS}\mbox{}\\
\textbf{Hit Dice:} 1d6 per necromancer level\\
\textbf{Hit Points at 1st Level:} 6 + your Constitution modifier\\
\textbf{Hit Points at Higher Levels:} 1d6 (or 4) + your  Constitution modifer per necromancer level after 1st

\paragraph{PROFICIENCIES}\mbox{}\\
\textbf{Armor:} Light armor\\
\textbf{Weapons:} Simple weapons, scythes\\
\textbf{Tools:} Embalming Tools\\
\textbf{Saving Throws:} Charisma, Constitution\\
\textbf{Skills:} Choose 2 from: Arcana, Deception, History, Insight, Intimidation, Medicine, Persuasion, and Religion

\subsubsection{Equipmet}
You start with the following equipment, in addition to the equipment granted by your background:
\begin{itemize}
\item (a) A scythe or (b) 2 daggers
\item (a) Leather armor or (b) padded armor
\item (a) A scholar's pack or (b) a explorer's pack
\item An arcane focus (spellcasting focus)
\item An embalming tools
\end{itemize}

\subsection{Control Undead}
At 1st level, you gain the ability to bring wild undead under your control by force. As an action, target an uncontrolled undead with an Intelligence lower than 8 within 30 feet of you. The undead makes a Charisma save DC 8 + your proficiency bonus + your Charisma modifier, and if it fails it is brought under your control indefinitely, but if it succeeds you cannot use this feature on it again for 24 hours. You may only control undead in this way that do not have a CR greater than your level. When you take over undead in this way, they are considered animated by you, and if you do not already control the undead you take damage equal to 1 per 1/8 CR of undead controlled (minimum 1) and your HP maximum is reduced by an equal amount. This Damage and HP reduction cannot be reduced by any means. Your hit point maximum is restored as the controlled undead die or leave your control, but you are not healed. You may control a maximum number of Undead equal to your Charisma modifier (minimum 1) times your Necromancer level.

You may control any undead you control as an action on your turn. At 9th level, this changes to a bonus action. If you are not actively controlling your undead, they will attempt to execute your last orders to the best of their ability. You only maintain control of your undead as long as they are within 1 mile of you. If they leave a 1 mile radius, they leave your control and act as normal undead.
Undead you control report to you psychically any creatures or environment that they can see. You innately know the general direction and distance of all controlled undead.

\subsection{Raise Undead}
Starting at 1st level, you may use your own life force to animate recently dead corpses. Using your action, you may touch a creature that died in the last minute and raise it as an undead under your control indefinitely.  You must spend an hour raising a creature if it has been dead for longer than a minute. Whenever you use this class feature, you take damage equal to 1 per 1/8 CR of undead controlled (minimum 1) and your HP maximum is reduced by an equal amount. This Damage and HP reduction cannot be reduced by any means. Your hit point maximum is restored as the controlled undead die or leave your control, but you are not healed. You may only raise undead in this way that do not have a CR greater than your level, and you are limited to undead with an intelligence lower that 8. At 9th level, you gain the ability to create undead with an intelligence score of 8 or higher, but it must be less than 13. At 18th level, you no longer have restrictions on the kinds of undead you can create based on intelligence score.

\subsection{Lifetap}
At 2nd level, you learn to freely manipulate your own life energy. As an action, you may touch any undead you have animated and deal any amount of damage to it up to it’s current health, giving you temporary hit points or healing any other undead you can touch that you have animated for an amount equal to the damage dealt. Alternatively, you can use an action to touch any creature and channel your own life force into it. This damages you for any amount up to half your maximum hit points or your current hit points, whichever is lower, that you choose and then heals the target for an equal amount.

\subsection{Necromantic Aspiration}
At 3rd level choose a Necromantic Aspiration: Caretaker, Reaper, and Summoner, all detailed at the end of the class description. Your choice grants you features at 3rd level, and again at 6th, 10th, and 14th level.

\subsection{Ability Score Improvement}
When you reach 4th level, and again at 8th, 12th, 16th, and 19th level, you can increase one ability score of your choice by 2, or you can increase two ability scores of your choice by 1. As normal, you can’t increase an ability score above 20 using this feature.

\subsection{Charnel Touch}
At 5th level, you learn to touch the lives of your foes and steal them for yourself. As an action, you may make a ranged spell attack (proficiency + Charisma modifier) against a living creature within 60 feet. If the attack hits, the target takes 1d8 damage plus your charisma modifier. This damage increases to 2d8 at 11th level, and to 3d8 at 17th level.
Additionally, if the target of the attack dies, you gain temporary hit points equal to half the damage dealt.

\subsection{Aura of Undeath}
Starting at 5th level, you gain an aura of undeath. You may use a bonus action to activate any or all auras you know and you may turn any of them off at any time for free on your turn.
You choose one of the following auras when you gain this feature, and again at 11th and 17th levels. When your Aura of Undeath is turned on you maintain any auras you choose until you dismiss the effect. You may use a bonus action on future turns to activate additional Aura effects, but may dismiss them for free. Your Aura of Undeath has a radius of 30 ft, and affects all undead you control within it. Some Aura effects require you to take damage at the beginning of your turns to maintain them. This cost is noted on each Aura effect.

If an Unholy Aura and a Holy Aura overlap, creatures in the overlapping areas get none of the benefits of either (so Aura of Undeath and a Paladin’s Devotion Aura would cancel out).\\\\
\textbf{Aura of Ferocity}\\
Costs 2 hp/round.\\
Affected creatures may add your Charisma modifier to their damage rolls with weapon attacks.\\\\
\textbf{Aura of Resilience}\\
Costs 1 hp/round.\\
Affected creatures may add your Charisma modifier to any saving throws they make.\\\\
\textbf{Aura of Retaliation}\\
Costs 2 hp/round.\\
Affected creatures may make an attack of opportunity against any creature that attacks them with a melee weapon or melee spell attack.\\\\
\textbf{Aura of Tenacity}\\
Costs 3 hp/round.\\
Affected creatures take less damage from non magical bludgeoning, piercing, and slashing damage equal to half your Charisma Modifier, rounded up (minimum 1).\\\\
\textbf{Aura of Terror}\\
Costs 2 hp/round.\\
Affected creatures become more menacing. Any enemy creature that starts its turn or moves within 5 feet of an affected creature must make a wisdom saving throw against DC 8 + your proficiency bonus + your charisma modifier or become frightened for 1 minute. They may repeat this save at the end of each of their turns. If they succeed on their save, they become immune to this effect for 24 hours. Any creatures that are frightened when this aura ends stop being frightened.

\subsection{Dark Nova}
At 7th level, you gain the ability to channel your life force to damage nearby enemies. As an action, you release Dark energy in a 10 ft. radius sphere with yourself as the point of origin. Creatures hit by this burst of dark energy may make a Constitution saving throw DC = 8 + your proficiency bonus + your Charisma modifier. On a failure they take 3d8 necrotic damage and are pushed 5 feet away from you. On a successful save, they take half damage and are not pushed. This damage increases to 5d8 at 11th level, and to 7d8 at 17th level. You may use this ability a number of times per long rest equal to your Charisma Modifier.

\subsection{Sacrifice Undead}
Starting at 10th level, you gain the ability to sacrifice your controlled undead to restore your life when you would be knocked unconscious. As a reaction to taking damage that would reduce you to 0 hit points, you may sacrifice an undead you control within 30 feet to instead drop to 1 hit point. You may use this feature once per long rest.

\subsection{Dominate Undead}
At 13th level, the abilities of your Control undead feature extend to intelligent undead and undead controlled by other necromancers. Intelligent undead are harder to control in this way. If the target has an Intelligence of 8 or higher, it has advantage on the saving throw. If it fails the saving throw and has an Intelligence of 12 or higher, it can repeat the saving throw at the end of every hour until it succeeds and breaks free.
If an undead you are trying to control is controlled by someone else, you may instead use your action to initiate a contest against said creature. Both of you roll 1d20 and add your Proficiency bonus and Charisma modifiers. If you win the contest, the undead is brought under your control, but in the event of a tie or if you lose, nothing happens. Once an undead has been contested like this, it cannot be contested again for 1 hour. If the undead has an Intelligence of 8 or higher, it may grant advantage in this contest to either necromancer.

\subsection{Ethereal Mind}
Starting at 15th level, your knowledge of necromancy allows you to understand concepts of spirituality foreign to normal people. First, once per day, you may use your action to see 60 feet into the ethereal plane for 10 minutes.
Additionally, once per day, when touching a dead body, you may begin a 1 hour ritual, during which you may converse freely with the soul that previously inhabited it, provided the soul is willing. If you have a possession of the spirit you are trying to contact, you may also use that to contact them

\subsection{Death General}
At 20th level, you gain the ability to choose a single controlled undead as your Death General. This undead can be chosen from any undead you control and gains a number of additional benefits.
Creating a Death General requires performing an 8 hour ritual every day for a week and 5,000 gp worth of materials.
\begin{itemize}
\item Your General’s HP cost is increased by 20 HP.
\item Your General’s Hit Points are their normal HP or 100, whichever is higher.
\item Your General’s Intelligence, Wisdom, and Charisma are replaced with your own, and they add your weapon and armor proficiencies to their own.
\item Your general has a control radius equal to your own and you may control undead within that radius. Your General can be controlled as long as you exist on the same plane of existence, and he will act to reach you if you are separated in such a way.
\end{itemize}
When you manifest your Aura of Undeath, Control Undead, or Dark Nova class features, you may do so from your General’s location, but they do not manifest from your own location if you do this.
You may use an action to begin seeing through your General’s senses. This lasts until you end it and causes you to become blind and deaf to anything around your own body.
You innately know the general direction and distance of any undead within the General’s control radius, and if an undead is within both yours and your Generals radius you know it’s exact location.
Any time you would die for any reason. Your General dies instead, and your body is teleported to a safe place chosen by your DM on the same plane of existence, unconscious but stable. Any undead you control leave your influence and become wild.\\\\
Alternatively, you may choose to become the Death General yourself. Becoming a Death General requires performing an 8 hour ritual every day for a week, but has no gold cost. When you do this, you gain the following features.

\begin{itemize}
\item Your Maximum HP is increased by 40, but these hit points can not be used to control undead.
\item You gain proficiency in Wisdom saving throws.
\item Your Charnel Touch deals an additional 1d8 damage.
\item The ranges of your Dark Nova and Aura of Undeath are doubled.
\item Your undead control radius is doubled.
\end{itemize}

\subsection{Necromantic Aspirations}
Necromancers share their affinity with undead, but how they treat their undead often varies. A Necromancer that minds their undead carefully is very different from a Necromancer that sits back while his army wages war, and both of these are very different from a Necromancer that fights on the front line with their undead as their leader.

\subsection{Caretaker}
Care for and enhance a small group of undead.

\subsubsection{Enhanced Animation}
At 3rd level, the damage and max HP cost per CR of animating undead increases by your proficiency bonus. As a result, undead you control may add your proficiency bonus to hit and have bonus HP equal to your necromancer level.

\subsubsection{Enhanced Resilience}
Starting at 6th level, undead you create have AC = 8 + your proficiency bonus + their dexterity modifier.

\subsubsection{Conscious Animations}
Starting at 10th level, your animations may maintain a facet of their living consciousness. These undead are capable of making their own decision, and have an intelligence and wisdom no lower than 10.

\subsubsection{Dark Restoration}
Starting at 10th level, your Dark Nova feature heals undead you control for an amount equal to half the damage they would have taken.

\subsubsection{Selective Binding}
Starting at 14th level, you gain the ability to animate bodies by binding willing spirits to them. You must spend an hour binding the spirit to the body, allowing the spirit to animate it. The ghost maintains the ability to leave the body, ending this effect. The animated body acts as if it were a normal undead, but is controlled by the spirit.


\subsection{Reaper}
Cut down your enemies with your own unholy powers.

\subsubsection{Bonus Proficiencies}
At 3rd level, you gain proficiency in Medium Armor and Martial Weapons.

\subsubsection{Dark Strike}
Starting at 3rd level, your melee weapon attacks deal bonus necrotic damage equal to your charisma modifier (minimum 1) once per turn. When you deal damage with this feature, you are healed for an amount equal to half your bonus damage rounded up (minimum 1).
At 10th level, your Dark Strike deals an additional 1d8 necrotic damage. This increases to 2d8 at 14th level.

\subsubsection{Extra Attack}
Starting at 6th level, you may make an additional attack when you take the attack action.
\subsubsection{Warcaster}
At 10th level you gain the ability to interweave your weapon attacks and your magic. When you take the Attack action on your turn, you may cast Charnel Touch or Dark Nova as a bonus action.
\subsubsection{Soul Reaper}
Starting at 14th level, whenever you kill a creature with Dark Strike, you may use your Raise Undead Class Feature on the creature you killed as a bonus action.


\subsection{Summoner}
Summon massive hordes of undead to crush your enemies.

\subsubsection{Efficient Animation}
At 3rd level, the damage and max HP cost per CR of animating undead decreases by your proficiency bonus when creating or controlling undead of a CR less than or equal to half your level (rounded up) or lower.

\subsubsection{Summon Undead}
Starting at 6th level, you no longer require a corpse to create undead, as you can just as easily summon them to a space within 5 feet of you directly from another plane such at the Shadowfell or the Negative Energy plane. This process takes 1 hour.

\subsubsection{Grandmaster}
Starting at 10th level, you gain the ability to control your undead from up to 10 miles away. This increases to 20 miles at 20th level.

\subsubsection{Unholy Siegemaster}
At 14th level, you gain the ability to create undead siege engines. These siege creatures function exactly as if they were normal siege equipment, but are considered controlled undead with a CR of 1 per 10 HP they have (so a Mangonel with 100 hp would be considered CR 10, or 75 hp is CR 7). Each siege engine takes 1 hour per CR to create and does not require a crew to operate.

\chapter{Chapter 4: Extra Rules}

\subsection{Familiars}

Familiars are animals or other beeings, that choose to follow the Character in game. A familiar is not bound in a physical form and therefor nothing more than a friend, that helps and gets help by a Character.

A familiar can be encountered during the adventure. It will stay by the side of the Character as long as it feels close to the Character.

In prinzipell, a Player can try to get any creature to be a familiar by roleplaying. But note, that some familiars might pursue a certain goal, such as freeing a friend or killing an arch enemey.

\subsubsection{Gameplay}

Between the "owner" and the familiar a bond formes over time. This bond is represented by the current \textit{Companion Level} (CL). The \textit{CL} rises over time. Whilst at the beginning, the bond to the Familiar is weak and both the owner and familiar do not realy understand each other. Slang, the Owner or the Familiar use might not be understood by the other side. Actions might be misunderstood or out right ignored.

The \textit{CL} rises with each fight (1/5 [20\%] of the XP for each kill the familiar was involved with). Also, if the Familiar and the Owner have "special achievements", like creative solutions to problems, involving the Familiar, you might be granted XP for the \textit{CL} by the GM.

\subsubsection{Fight with an Familiar}

Tho Familiars are most likely not verry strong or brave, a Familiar will also fight if needed.

Within the fight a familiar might execute one of 5 actions, that the Player may choose \textbf{in addition to} its normal actions. This new action is called a \textbf{Familiar-Action}.
Using the \textbf{Familiar-Action} does not count against the normaly available Action or Bonus-Action.

For example: A level 4 Pladin has a street-cat as a familiar. During the battle he does the following: Attack (Action), heal (Bonus-Action) and commands the familiar to hide (Familiar-Action).

The Familiars action is executed immediately after the Players turn, so it's initiative-turn is after the players one. The actions you can command a familiar to do, are one of the following (choose on):

\begin{itemize}
\item \textbf{Distract} \\ The familiar flies/runs towards a choosen Creature to distract it. Any following attack by the distracted Creature has disadvantage and the Creature has disadvantage on the next Check/Saving Throw. To distract, the familiar has to enter the distracted Creature space and may provoke a oppotunity attack of the distracted Creature if leaving the space again.
\item \textbf{Support} \\ This Familiar-Action will manisfest in different ways. A rat may bite the ankle of an creature attacked by the Player or crawl over the attacking arm and bite the attacked creature as the Character attacks. The Character has advantage on the next attack role. To support, the familiar has to enter the attacking creatures space and may provoke an opportunity attack if leaving the space again.
\item \textbf{Attack} \\ The familiar attacks all on its own the possibly giant opponent. The Player chooses the attacking-action (of the familiars possible attacking Actions). Regardless of the familiars remaining speed, it returns to its owner after attacking without provoking an opportunity attack. Afterwards it cannot move anymore.
\item \textbf{Interact} \\ The Familiar may interact with any object, to fullfill any task. It has to be able to interact with the object, to fullfill the task. Ask you DM, wether or not an choosen Action is legit or not and an Familiar can interact with the chossen object. For example, an weak owl might not be able to interact with an old, rusty switch.
\item \textbf{Hide} \\ The familiar may try to hidde in a certain spot. For that, he makes an stealth check against the opponents passiv perception, which is closest to the familliar. Successfull or not, the familiar will go to the choosen spot.
\end{itemize}

Additionally to the above actions, an familiar might be commanded to move up to its speed.

If the Companion HP falls to 0, it becomes seriously injured and unconscious. It then cannot be commanded any more and checks every round for whether or not it will become stable using a plain d20 without modifiers. If it rolles 10 or higher, it gains succeeds one time, other wise the role failes. If it succeeds 3 times it will become stable but stay unconscious for 1d4 hours. If it fails 3 times, it is dead. Rolling a 20 will give it 1 HP but it will only be able to use the hide action untill the fight finishes. Rolling a 1 will give it 2 fails.

To help stableize the familiar, any Player may use its action to make a DC10 Animal Handling (WIS) check. On success, the Familiar will be stable but stay unconscious for 1d4 hours.

If an Familiar becomes seriously injured during a fight, it makes a number of friendship checks (dependend on the \textit{CL}) to see wether or not it will stay with its owner after this traumatic experience. It has to make a d20 check without modifiers and role higher than 16 - \textit{CL}. So at first level, you have to role 15 or higher, at third level you have to role 13 or higher and at 5th level, you have to role a 11 or higher.

Regardless of the outcome, it will take 1d4 downtime for the Familiar to recover from its wounds. If it stays with its owner, the owner has to take the downtime with its Familiar.

\newpage

\subsection{Castle Management And Administration}

\end{document}