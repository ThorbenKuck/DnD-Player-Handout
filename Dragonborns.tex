\paragraph{Origin}

Nobrut wurde in der Vergangenheit von Drachen und Drachen-Geborenen beheimatet. Die Hauptstadt der Drachengeborenen hieß Drakor. Auch wenn Drakor heute nur noch aus einige Ruinen besteht, welche aus dem Sand Gombruts ragen, so war es einst eine Pompöse Stadt, an der alle Wesen willkommen waren. 

Als erstes verschwanden die Drachen. Die Drachengeborenen lebten friedlich in der Mitte von Nobrut und bekamen dieses nicht mit. Doch nach nur ein paar Jahren wurden auch diese Überrascht. Zu erst wurden die kleinen Dörfer, nahe der Goraba-Bergkette überrant. Die Gombrut, immer auf der Suche nach Nahrung überranten zu erst die kleineren Dörfer. Ein Krieg brach aus und das Obwohl die Drachengeborenen ein friedliches Volk waren. Die Gombrut hatten Angst vor den Drachengeborenen, da sie Nachfahren der mächtigen Drachen waren, doch die Wahrheit war, dass die Drachengeborenen ohne die mächtigen Drachen verloren waren. Die Drachen tauchten jedoch nicht wieder auf und die Drachengeborenen waren dem Untergang geweit. Selbst Versuche, ein Abkommen zwischen den Gombrut und den Drachengeborenen zu verabschieden scheiterten. Drakor wurde über einen Zeitraum von 9 Monaten belagert und beide Seite verloren viele Krieger und Kampf-Magier. Nach den 9 Monaten hielten die Drachengeborenen immer noch ihre Stellung, aber dennoch erschien die einzige Möglichkeit, wie die Drachengeborenen überleben konnten, der Rückzug zu sein. Die Mauern Drakor's fielen und der Herrscher Ayyaam, ging zusammen mit ihnen unter, während er zusammen mit einigen tapferen Soldaten, die Gombruts lange genug zurück hielt, damit sein Folk fliehen konnte.

Heute leben die Drachengeborenen verstreut über ganz Goobink, mal in kleineren Dörfern, mal in mitten von anderen Wesen. Eines teilen sie allerdings bis zu dem heutigen Tage: Einen Hass auf die Gombruts und ein Streben in ihrem Herzen, Drakor wieder zu erbauen und den Drachengeborenen wieder ihren rechtmäßigen Platz auf Norbut zurück zu geben.