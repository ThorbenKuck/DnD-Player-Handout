\chapter{Chapter 5: Rule Extentions}
\section{Familiars}

Familiars are animals or other beeings, that choose to follow the Character in game. A familiar is not bound in a physical form and therefor nothing more than a friend, that helps and gets help by a Character.

A familiar can be encountered during the adventure. It will stay by the side of the Character as long as it feels close to the Character.

In prinzipell, a Player can try to get any creature to be a familiar by roleplaying. But note, that some familiars might pursue a certain goal, such as freeing a friend or killing an arch enemey.

\subsection{Gameplay}

Between the "owner" and the familiar a bond formes over time. This bond is represented by the current \textit{Companion Level} (CL). The \textit{CL} rises over time. Whilst at the beginning, the bond to the Familiar is weak and both the owner and familiar do not realy understand each other. Slang, the Owner or the Familiar use might not be understood by the other side. Actions might be misunderstood or out right ignored.

The \textit{CL} rises with each fight (1/5 [20\%] of the XP for each kill the familiar was involved with). Also, if the Familiar and the Owner have "special achievements", like creative solutions to problems, involving the Familiar, you might be granted XP for the \textit{CL} by the GM.

\subsection{Fight with an Familiar}

Tho Familiars are most likely not verry strong or brave, a Familiar will also fight if needed.

Within the fight a familiar might execute one of 5 actions, that the Player may choose \textbf{in addition to} its normal actions. This new action is called a \textbf{Familiar-Action}.
Using the \textbf{Familiar-Action} does not count against the normaly available Action or Bonus-Action.

For example: A level 4 Pladin has a street-cat as a familiar. During the battle he does the following: Attack (Action), heal (Bonus-Action) and commands the familiar to hide (Familiar-Action).

\newpage

The Familiars action is executed immediately after the Players turn, so it's initiative-turn is after the players one. The actions you can command a familiar to do, are one of the following (choose on):

\begin{itemize}
\item \textbf{Distract} \\ The familiar flies/runs towards a choosen Creature to distract it. Any following attack by the distracted Creature has disadvantage and the Creature has disadvantage on the next Check/Saving Throw. To distract, the familiar has to enter the distracted Creature space and may provoke a oppotunity attack of the distracted Creature if leaving the space again.
\item \textbf{Support} \\ This Familiar-Action will manisfest in different ways. A rat may bite the ankle of an creature attacked by the Player or crawl over the attacking arm and bite the attacked creature as the Character attacks. The Character has advantage on the next attack role. To support, the familiar has to enter the attacking creatures space and may provoke an opportunity attack if leaving the space again.
\item \textbf{Attack} \\ The familiar attacks all on its own the possibly giant opponent. The Player chooses the attacking-action (of the familiars possible attacking Actions). Regardless of the familiars remaining speed, it returns to its owner after attacking without provoking an opportunity attack. Afterwards it cannot move anymore.
\item \textbf{Interact} \\ The Familiar may interact with any object, to fullfill any task. It has to be able to interact with the object, to fullfill the task. Ask you DM, wether or not an choosen Action is legit or not and an Familiar can interact with the chossen object. For example, an weak owl might not be able to interact with an old, rusty switch.
\item \textbf{Hide} \\ The familiar may try to hidde in a certain spot. For that, he makes an stealth check against the opponents passiv perception, which is closest to the familliar. Successfull or not, the familiar will go to the choosen spot.
\end{itemize}

Additionally to the above actions, an familiar might be commanded to move up to its speed.

If the Familiar HP falls to 0, it becomes seriously injured and unconscious. It then cannot be commanded any more and checks every round for whether or not it will become stable using a plain d20 without modifiers. If it rolles 10 or higher, it gains succeeds one time, other wise the role failes. If it succeeds 3 times it will become stable but stay unconscious for 1d4 hours. If it fails 3 times, it is dead. Rolling a 20 will give it 1 HP but it will only be able to use the hide action untill the fight finishes. Rolling a 1 will give it 2 fails.

To help stableize the familiar, any Player may use its action to make a DC10 Animal Handling (WIS) check. On success, the Familiar will be stable but stay unconscious for 1d4 hours.

If an Familiar becomes seriously injured during a fight, it makes a number of friendship checks (dependend on the \textit{CL}) to see wether or not it will stay with its owner after this traumatic experience. It has to make a d20 check without modifiers and role higher than 16 - \textit{CL}. So at first level, you have to role 15 or higher, at third level you have to role 13 or higher and at 5th level, you have to role a 11 or higher.

Regardless of the outcome, it will take 1d4 downtime for the Familiar to recover from its wounds. If it stays with its owner, the owner has to take the downtime with its Familiar.

\twocolumn[{%
\begin{@twocolumnfalse}
\header{The Familiar Table}
\begin{dndtable}[ l  l p{12cm} l ]
\textbf{\textit{Level}} & \textbf{\textit{Exp}} & \textbf{\textit{Description}} & \textbf{\textit{Friendship-Checks}} \\
1st & 0 & The Familiar can make any action with an successfull DC 15 Animalhandling Check. Communication with the Familiar is only very conditionally possible. To comunicate with the Familiar, you have to succeed on a DC20 Animalhandling Check.
Only easy "yes or no" questions like: "where there Creatures?" can be understood and answered. & 2\\
2nd & 50 & The Familiar can make any action with an successfull DC 10 Animalhandling Check. Communication with the Familiar is only very conditionally possible. To comunicate with the Familiar, you have to succeed on a DC15 Animalhandling Check. & 2\\
3rd & 100 & The Familiar can make any action with an successfull DC 5 Animalhandling Check. To comunicate with the Familiar, you have to succeed on a DC10 Animalhandling Check.
You may communicate using more complex questions like: "Where there Orcs", "How many Orcs did you see?" or "Is the way blocked by something that we can surpass?". & 3\\
4th & 300 &  To comunicate with the Familiar, you have to succeed on a DC5 Animalhandling Check. The trust between the Charakter and the Familiar is now big enough, that you do not have to role to command the Familiar to make any action. The Familiar now reports on its own about unusual or weird things it has seen. The Player does not have to question the Familiar about every little detail, but he/she still may. & 3\\
5th & 600 & Based upon the experience that the Player and the Familiar have, they understand eachother without the need of word. Fluent communication is possible and the familiar shares his deepest dreams/hopes/nightmares. At this point, the Familiar might report about hidden motives it may follow. Also, the Familiar rises 1 level. Ask your GM for that & 4\\
\end{dndtable}
\end{@twocolumnfalse}
}]

\newpage

\chapter{Chapter 6: New Rules}
\section{Castle Management And Administration}

If you happen to conquer/take over a castle, manor or any other shelter, you may use it as your headquarter or as an outpost.

Though you can have as many of those as you want, you may only have one headquarter at any time. Due to the sheer complexity of maintaining a whole city and the fakt that most headquarters are connected to a city, you are limited to one Headquarter. An exception to this rule is, to take over another Headquarter in war.

Note that, if you takeover an occupied castle, manor or any other shelter it is considered an act of war! This means, that the currently occupying Creatures and allys will be hostile towards you and your allys. Also, if you kill all present or just some/most Creatures and leave the rest to retreat, your newly occupied space will have a verry bad and brutal reputation, which will lead to difficulties in politics and negotiations between your Headquarter and other Headquarters and Outposts, espacially allys of the slaughtered Creatures.

\subsection{Basic Administration}

Basic administration, like what is needed to run an post if your out, slaying beasts in dangerous habitats, can be handeld by an representant.

For example: Your representant, that keeps the Headquarter up an running is the Buttler but he still is only your Buttler, not the bos of the other Personal. If you do not name anyone as your direct representant and have no rules in play, you might run into an anarchy like bahaviour of your employees. For example: even though your buttler tries his best to convince the Warlord that it is not a good idea to gut the Nobels, he does and declares war on a personal enemy of him, all whilst the Treasurer runs as fast as possible, his coat filled up with gold.

Make sure to have a basic ruleset up an running and someone in charge as an direct representant of you, the leader, that in the worst case can be used to set an example.

\subsection{Headquarter}

A headquarter is connected to a whole city. If you choose to not take a lot of downtime to handle and maintain the Headquarter and the connected city, you will have to hire personal that do so. Those consist of the following:

\subsubsection{Builder}

The Builder is responsible for building, repairing and extending Buildings within the city. His resources are versatile. He needs:\\
Stone, Wood, Clay, Man, Money, Tools, Metal and more.

The more experienced your Builder is, the better his time and resource management becomes. At the beginning, a new Builder might not understand the time a House needs to be build in contrast to a farm and how many resources either of those needs.

The more Money he has, the shorter he has to wait for certain resources (even man). He will try to get the money from the Treasurer. You may specify a budget for the Builder.

This means, that building takes longer, if your Builder is inexperienced. So make sure to have a good and experienced Builder inside of your Headquarter. The Builder is also responsible for managing repairs and buildings of Outposts.

Tho they need multiple imputed, which they will optain on their own over time, you can have only one Builder per Headquarter, expect if some work as a team.

\begin{paperbox}{Idleness as Ideal}
Nearly all Builders are lazy. They have People for every job. But contrair to their lazyness, they know how to build and how to manage stuff. Make sure that your Builder has enough People to ensure a fast build time.

Also be prepared that they will change their imputed often!
\end{paperbox}

\subsubsection{Buttler}

The Buttler is your direct respresentant inside of the Headquarter (not in the whole city). Since he/she is your representant inside of the Headquarter, he should have an personal interrest to keep costs as low as possible.

If no other work is asigned, he will do work like cleaning and tidy up or asuring that every personal is staying within the bounds of the castle. He does not need special interactions or resources. 

Because he does not need to interact with the Treasurer, he will probably have the lowest costs. The only costs that he will produce are costs which you specify, like sending bards to the near by towns or managing the wages for the construction workers.

A Buttler does not need and want employees. He is on a par with other employees with the twist that he may command them to do things. A Headquarter can have only one Buttler.

\begin{paperbox}{Your right hand}
Since the Buttler is your direct representant, you can make him do anything. 

Though he will use other resources (like other Personal), he is the best contact to maintaining the Headquarter if you are out, slaying Dragons in Dungeons.
\end{paperbox}

\subsubsection{Nobel}

The Nobels are responsible to establish new contacts, create new political relationships and espacially maintain those. They are your political representation. These relationships might be of any type like import/export of goods, military help or simple friendships and favors.

A good and experienced Nobel can deescalate difficult situations (like cold wars) and get low prizes for taxes on pass through of other territory, export-taxes or great deals on imports of goods from it.

If no Task is specified for the nobles, they will try to establish and maintain political friendships between Headquarters and therefore obtain new Political allys.

The more Money he has, the faster and more efficient he and his team get into other politicians domains and get favors, since money is key to most things in politics. Also, he will be able to hire better Nobels for Outposts. He will try to get the money from the Treasurer. You may specify a budget for the Nobels.

A Nobel may have its own employees to help with its tasks. Otherwise a Headquarter can have only two Nobels, expect if some work as a team.

\begin{paperbox}{A completely different world}
Nobles are snooty. Exclamation mark. But among their peers, they are within their element. 

Tell the Nobles what they should focus on and which citys they should visit. But also explain why. Make it a good explanation, so they feel flattered. This will give you the best chances of success.
\end{paperbox}

\subsubsection{Treasurer}

The Treasurer is maintaining the fortune of the Headquarter. He might be assigned to maintain the fortune of Outposts, but he has to be experienced to do so. If you is experienced, the Treasurer will maintain the money in the Headquarter and 2 Outpost, eliminating the need for multiple Treasurers. He will also be responsible to collect taxes, if any of those are asigned and no other Person is put in charge to collect those.

If no Tasks is specified for the Treasurer, he will maintain the fortune by restricting the access to the fortune by other personal to its own pain barrier of money.

You can specify a budget for everything and the Treasurer will try to match those. If you do not specify any budget at all, the Treasurer will act upon its own and choose budgets as he feels like (which might be good or bad!).

A Treasurer may have its own employees to help with its tasks. Otherwise a Headquarter can have only one Treasurer, expect if some work as a team.

\begin{paperbox}{A land of Restrictions}
The Treasurer will maintain your fortune. This means, he will decide for you, which section will receive how much budget. 

Tell him what he should focus on, to make sure, he does exactly spend the money as you whish and not as he feels.
\end{paperbox}

\subsubsection{Warlord}

A Warlord is responsible for maintaining security within the Headquarter. This might be security for any kind of thread (internall, external, terrorism, ...).

Most of the time, a Warlord will try to establish a new army, which he will do by default. Since a Warlord is alone at the beginning, he will need help by starting out. If nothing else is specified, he will build up an self sustaining Army, which does only need Money. If you establish an War Post, the Warlord of the Headquarter will be the direct supervisor of the Warlord inside the War Post.

The more Money he has, the faster and stronger his army gets over time. He will try to get the money from the Treasurer. You may specify a budget for the Army.

A Warlord needs imputed, which he will try to optain on his own. But you may only have one Warlord per Headquarter, expect if some work as a team.

\begin{paperbox}{Strength through Strands}
Most Warlords are strict and direct and that is good. If you give your Warlord enough resources and let him just do his stuff, he will most likely succeed on building an army. 

Just make clear, where the line is, that he should not cross. Else he will build up an army state.
\end{paperbox}

\begin{commentbox}{Other}
You might as well have any other job offer you like.

Talk to the GM and other players, if you want to have other available posts/personal and espacially about what it should handle and maintain.

A recommendation is to make those optional. If you make a new mandatorry Personal, you will have to adjust everry little detail of this, to integrate the new Role.
\end{commentbox}


\subsection{Outposts}

If you want, you can have multiple outposts. An Outpost can be accquired by occupy a suited space like an cave system or a plain with a forrest. An Outpost can only be build and maintained, once you established an Headquarter.

You will than have to build up an Outpost. The requirements depend soly on what kind of outpost you want:

\subsubsection{Trading Post}

A Trading Post is either a Outpost, receiving and sending goods from and to your Headquarter or a producer of a certain good, like Wood or a change station, taking goods you don't need and trading to collect goods you need. You may trade with any ally you like. Over time, if your Trading Post gets a better and better reputation, you will get better deals.

An Trading Outpost is basically a house, surrounded by a smal wood-fence with smaller houses around the main house. It has a storage house and multiple living quarters for the Workers, as well as a better house, holding the board members. The main house is the main spot, where traiding happens.

A Trading Post is considered claimed territory, which means that you may enforce taxes on passing traders, based upon the goods they wear or even on travelers, just because the pass through.

Requirements:

\paragraph{Resources}
\begin{itemize}
\item Wood
\item Stone
\item Clay
\item 200 man hours
\end{itemize}

\paragraph{Man}
\begin{itemize}
\item \textbf{Nobel (board member)} A Nobel, that is running the place and responsible for traiding and deals.
\item \textbf{Treasurer (board member)} A Treasurer to maintain the money.
\item \textbf{Other Personal} Wi Building buildings, ll be hired by the Nobel. Needed to organize/harvest the goods.
\end{itemize}

\paragraph{Territory}

The Territory required to build a Trading Post is central and open with as much roads leading from and to other Citys as possible.

\subsubsection{War Post}

A War Post is a highly armed Outpost that is ready to react to any kind of war like behaviour or simply attack any other city. The War Post is ready to fight at any time, which in fact means, you may command an War Post to attack any target you like. Over time, as your reputation becomes better, you may use an War Post to hold soldiers of other allys or send your soldiers to War Posts of allys.

An War Post looks basically like you would expect it to. You have a big main tend, most likely in the middle of the war post containing the board members. This main tend is surrounded by multiple smaler tends, containing multiple soldiers each. You may find training grounds or similiar things and the whole Outpost ist surrounded by a bigger fence of wood, with little spikes in front. Also, you might find some look-out towers.

A War Post always looks robust, but ready to be dismantled fastly, if you know what to do. To dismantle a War Post takes half as long as to build it up. You loose 20\% of the resources put into the War Post to wear and tear.

A War Post is considered claimed territory, which means that you may enforce taxes on passing traders, based upon the goods they wear or even travelers, just because the pass through.

If you have many War Posts, you may be considered a Threat by other Head Quarters. Keep in mind, that they then might attack you or your outposts to reduce your overall miltitary presence or they even might declare war against you. A good set of experienced Nobels is mostly enough to justify War Posts.

Requirements:

\paragraph{Resources}
\begin{itemize}
\item Wood
\item Stone
\item Clay
\item Weapons
\item 130 man hours
\end{itemize}

\paragraph{Man}
\begin{itemize}
\item \textbf{Warlord (board member)} A Warlord, running the War Post and commanding the units based upon the commands he receives.
\item \textbf{Other Personal} Will be hired by the Warlord, mostly military personal. They bring multiple functions with them.
\end{itemize}

\paragraph{Territory}

The Territory required to build a War Post is not realy defined. It is an tactical environment, open to see enemys early and closed to not be seen as fastly.

\begin{commentbox}{Other Outposts}
Talk to your DM, to establish other outpost. Define their function and what should be done within them as well as what is required to build and run this Outpost.
\end{commentbox}

\subsection{Personal}

To get Personal is easy. Send out some bards to ask around or spread some rumors in multiple taverns. The better your offer, the more likely you'll get more capable Personal. Note that the Personal has to have an office in your Headquarter. This might be a room in a castle or a luxurious house near the inner ring of your city.

Once you got the word out there, you will find some candidates. Those candidates can be interviewed and hired by anyone you like, but more successfull is to interview them your self to find out the different trades of that person.

Every Person, that operates an office has different trades. They manifest in the following:

\subparagraph{Pro} This is the speciality of that Person. It is what makes the Person specail and stand out over others in certain situations.

\subparagraph{Con} This is some flaw, a tic, a quirk or anything like this, that may cause trouble in certain situations.

\subparagraph{Type} A Person is of a certain Type, which defines his speciality within his Job.

\subparagraph{Factor} The factor states, how this Person influences the Job he has to do.

\subparagraph{Name} This is how the Person wants to be called.

\subparagraph{Race} Of course, a Person is of a certain Race. Based upon the history of the given Race and the shared Ideals of that Race the Person might behave differently in certain situations.\\

Once you hired someone, you can as well fire them again. If, for example, your Treasurer is a bad apple, you may throw him hinto the mud by the piggs. But you will have to find a replacement fast. You can imagine that, if you fire the Treasurer, Warlord, Builder of the Nobel, you have a running machine without gears.

Have a replacement up your sleeve, if you plan to fire someone and be sure that he keeps his mouth shut about secrets that you do not want to hear outside of you walls, like the position of your treasury or your bedroom.

Because every person prefferes something differend, it is hard to state, what a hiree wants. Maybe your new Treasurer wants to live inside your big castle or inside of an luxurious house. Maybe your new Warlord want's a whole wing of your castle, or just a tower. Ask them what they want, to ensure that you will have the right thing for the right person.

\subsection{Other Personal}

Other Personal, such as construction worker, state farmer or so on, will have wages. To make gameplay faster and easyer, they are asumed as an average. So if you have 20 construction workers in real life, you have some that are better, some that are faster, and so on. This would complicate everything. Therefore all skills of those 20 construction workers will be be treatet as "average" across all 20 workers.

So, lets say, an Worker takes a wage of 1 Gold per day and are promised a day ration every day. The wages are due every 20 Days. And you have 20 Workers per day. This means you will have to provide 20 day rations per day and 400 Gold every 20 days. You might of course change the day of paying out in your rules, the wages by your Builder. Also, you might have a better deal, if you ask a Noble to reach out to any ally and ask for a cheap deal for the Day rations.

In prinziple you are able to make such deals on your own. But keep in mind, that such deals are long days of travel, diplomatic discussions and demanding multiple favors.

Also, it is important to realize that not everything is free of accidents. Make sure, that you are prepared for those as much as possible! Maybe a Building collapses, a worker is critically injured or an enemy army is trying to take over the city. Do not over favor one or another section of your Headquarter.

\subsection{Buildings}

You can have multiple Buildings in certain Places. In theory you can build every Building in any Headquarter or Outpost. However, because Outpost are limmited in theire reachablity, you cannot simply build every Building within every Outpost.

Following are some Buildings and requirements/costs that might be of an inspriation for what to Build:

\begin{itemize}
\item \textbf{House (1 Family, 4 Rooms)} \\ A House, containing 4 rooms requires wood, stone, clay and man to be build. The Materials might be of lower quality. It takes 48 man hours to complete. Idle it does not take up resources.
\item \textbf{Luxury House (1 Family, 7 Rooms)} \\ A Luxury House, containing 7 romms requires wood, stone, clay, metal and man to be build. The Materials must be of moderat quality. It takes 52 man hours to complete. Idle it does not take up resources.
\item \textbf{Farm (Animals)} \\ A Farm conatining animals requires wood, stone, clay and man to be build. It takes 60 man hours to complete. Further, it needs 1 pound wheat per cow to feed the animals. It produces either meat, eggs, livestock or anything else that suits the foot presentet. It requires a plain land to be build on.
\item \textbf{Farm (Crops)} \\ A Farm conatining crops requires wood, stone, clay and man to be build. It takes 60 man hours to complete. Further, it needs seed to build up the crops. It produces non-meat eating equivalent to the seeds planted. It requires a plain land to be build on.
\item \textbf{Tavern} \\ A Tavern requires wood, stone, clay and man to be build. It takes 60 man hours to complete. Further it requires a tavern ceeper to run the Tavern and it further requires a steady instream of food and drinks to be sold. A Tavern is best build in an central spot of your city.
\item \textbf{Quarry} \\ A Quarry requires wood, stone, clay, metal and man to be build. It takes 90 man hours to complete. Further it requires an elite worker and multiple workers to run the quarry. It produces stone, raw metal and clay. It requires a mountain, which it can safely work on.
\item \textbf{Forge} \\ A Forge requires wood, stone, clay, metal and man to be build. It takes 100 man hours to complete. Furhter it requires an smith and raw metal to run the Forge. It produces Metal and on request armor and weapons. For that it may requires other materials.
\item \textbf{Grocery Store} \\ A Grocery Store requires wood, stone, clay and man to be build. It takes 48 man hours to complete. Further it requires an owner and a steady instream of grocerys to be sold to run the Grocery Store.
\item \textbf{Stables} \\ A Stable requires wood, stone, clay, metal and man to be build. It takes 55 hours to complete. Further it requires an stable keeper and animals to be run. For each animal to keep, 1 pund of wheat is required per day.
\item \textbf{Mill} \\ A Mill requires wood, clay, metal and man to be build. It takes 65 man hours to complete. Further it requires an miller and a steady instream of crops to be run. It produces Wheat and other cereals. A Mill requires a place, exactly close to a river.
\item \textbf{Sawmill} \\ A Sawmill requires wood, clay, metal and man to be build. It takes 68 hours to complete. Furhter is requires an owner and a steady instream of raw wood to be run. It produces wood.
\item \textbf{Woodcutting Cottage} \\ A Woodcutting Cottage requires wood, clay, metal and man to be build. It takes 60 man hours to complete. Further it requires an owner and a location close to the wood to be run. It produces raw wood.
\end{itemize}

If you have all requirements to build a building, you can asign workers at the Builder to do those jobs.\\
You may use localy produced or imported goods to build a building. You can reesign any number of workers at anytime to any building at the Builder.

The Builder directly influences the time and resources a building requires. Whilst the time is influenced either positiv or negative based on the factor of the Builder, the resources might only be influenced negatively by an unexperienced Builder. If your Builder is inexperienced, it might mis calculate the requirements or falsely asign resources.

Note that building a building may complete without complications and nothing may ever happen. But randomly a building may get damaged or even collapse over time. But, a building may not complete without complication. Complications and damages are expressed in man hours to repair the damages. Materials to repair damages are declared seperatly.

\subsection{Population}

The Population, living within the city of you Headquarter has an overall feeling, expressed as a \textbf{Mood}. The mood is influenced by (not exclusively):
\begin{itemize}
\item To strict rules / Not enough rules
\item To much/less military present
\item Not enough resources (like: no Tavern, Grocery Store, Mill, ...)
\item Not enough other people around them
\item Not enough jobs (Sawmill, mill, Woodcutting Cottage, Forge, Quarry, ...)
\item To high taxes / To little money for roads or other works.
\end{itemize}

You should find the right balance. Without taxes, you will have trouble building new Buildings, fixing broken stuff but with to high taxes, People will run.

Also, if you have to little resources and no rules in play, people might fall into an criminal rage or even revolt. 

If your Population revolts, you have a real Problem. With an ongoing revolution, you will loose allys and deals with them. Your overall reputation will sink fastly. At worst, with to little military in play, they will takeover the city and establish a new leadership, not stopping at blood. Take into account, that if a rebell cell is forming, you should fastly do something against it.

\subsection{Taxes}

Taxes are a way of getting Money. A direct but an ligitimate one. You may have taxes collected by your Treasurer or by your Nobels (which most likely will hire someone to do so). Ofcourse you might also have your Warlord collect the Taxes, but he will enforce the Taxes and be more direct.

\subsection{Import/Export}

Once your Nobels have established an certain reputations in other citys, you can start to import/export goods.

Those goods may be everything. You can have wood importet from a woodcutter vilage or stone from a quarry city. The only important things are:

\begin{itemize}
\item \textbf{Traders} Hire traders to drive goods from city a to city b, or bring them yourself, if you have the time.
\item \textbf{Taxes} Import and export taxes are taxes that either the receiving city or you enforce on goods that enter the city. It is a source of income, but make those taxes to high and you will have problems getting goods!
\item \textbf{Routes} If you have no good routes, leading from city a to city b, you will leave the routes to the traders. This means, they might enforce costs, to compensate taxes of traversing other people territorys.
\end{itemize}

Import is a good way of ensuring a stead stream of goods, if you have enough money to get those and enough political influence to get good deals of that.

\subsection{Rules}

You decide the rules of your Headquarter. And those are inportant! By default, there are no rules. People will live whereever, take whatever and military will only hear at the Warlord.

To prevent a military dictatorship and other similiar bad states, you should implement a set of rules, that describe certain situations. A rule set may be simple (like 4, 5 rules) or it may be whole book. It realy depends on the Players.

But note! People are mean! If they find a loopwhole, they will take it. They might find a way to get money for nothing or a way to take over the whole city.

For that, you should think about what your city should become. Should it be an Monarchy with you at the top? Or should it be a Democracy, always concerned about what the people living within the city think? Should its rules be finite and fixed or endless and open? Every selection has both up and downsides. For example:\\
If you choose to make an Democrazy with an finite amount, but changable rules, most likely people will come more likely. But also, you will require more people to administrate the whole process, which in terms needs more money, which either has to come from the people or from exports.\\
But if you choose a Monarchy with a smal, finite and not changable amount of rules, you will have the decision making in everything, but people will maybe not fell as great in your city and you have to look more often into your city to see that it does not go down hill.\\

Your ruleset may be as pronounced as you like. But do not make it to restrictive. If your ruleset is to restrictive, people will not like you.

By default, your city has no polity, therefor it is sort of an Anarchy

Here are some Politys, you may inplement within your city. If you have problems finding a Polity, you may roll on the following table or simply choose your Polity:

\begin{dndtable}[cX]
\textbf{d6} & \textbf{Polity} \\
1 & Direct Democracy \\
2 & Representative Democracy \\
3 & Monarchy \\
4 & Dictatorship \\
5 & Republic \\
6 & Marxism
\end{dndtable}

Thos are not all politys you can use. It all depends on your ruleset and how you imagine your city to function.

Another alternative is, to leave even this part, creating and mainting a set of rules to your right hand, the Buttler or to anyone realy. You can name anyone as a Leader of your city. But be sure, that he is not planing something behind the hand. Making deals with your opponents to overthrow your state and become the owner itself.

\subsection{War}

War is simple and easy. 2 or more fronts try to kill each other. It is bloody and brutal and to be avoided at any cost! Declaring war is as easy as sending a bard or any other messanger to the one you want to declare war on. As the messanger reaches the other party, war breaks loose. Another way of declaring War is to simply carry out a war by attacking your oppont with your army or alone, for all that matters.

In War there are no rules, except for the rules within your kingdom. Be prepared to fight at any given moment. Be prepared that assassins might infiltrade your Head Quarter to kill the head of your Kingdom (you) or terrorists killing residents of your Head Quarters city. You may do the same, but do not expect understanding of your actions. Sympathy within the opponents space is rare and absolutly needed to dismantle the other party from within

If this is not your stile, to run as many intrigues as possible, you may simply kill the opponent. This is bloody and takes up most resources. Residents of your citys will not like it unless you win and allys might be disappointed in you for simply killing your opponent. However, a declared War only ends if:

\subsubsection{Either side surrenders}

This is the more elegant and more resource saving way. Maybe your intrigued so long, that most people are leaving your opponent, or you spread rumors and/or helped forming a rebel-cell, letting residents of your opponents city rebell against their leader.

Get your opponent to surrender and sign a contract, giving you want you want. This may take some time to discuss, but in the end not many resources will be consumed to get the best result.

The drawback is, that you will have the other party still alive. They may plan something behind your back. But they might as well become your ally.

On the other hand, surrendering will most likely give you a way out of war, without dying. Even though, you most likely will loose your land, most of your army, head of state and other personal will survive. Well, some at least. If you negotiate good. Hopefully.

\subsubsection{The Leader of either side dies} 

This is the bloody way. You use all your available resource, hireing assins, attacking with your army or killing traders from and to the city of your opponent. Since Weapons need care, soldiers need food and damaged buildings need to be repaired, this way takes a lot of resources.

To fullfill the needs of those resources, you might ask allys for help, or plunder other citys. The last one will let your reputation sink but give you a cheap instream of resources.

Though there are no rules, killing women and children in a blood rush or stealing livestock and/or food from civiliance is giving you a bad reputation. This should be clear from the get go. Allys might turn their back on you or out right join your opponent, even though this is the cheapest way of getting more resources.

Once the leader of the opponents side is dead, you may claim their thrown. You expand your kingdom by a second Head Quarter. The second one will be lead by the Army at first untill you establish a new head of state. Since this is new territory is now considered claimed territory, your rules apply to the new kingdom aswell. This might not be liked by all residents of this newly conquered city. Prevent the forming of new rebell cells as soon as they start to form.

On the other hand, this way of loosing is bad. You will die. If you executed to many acts of war, surrender is not an option any more. The opponents will hunt you, until they or you die. Maybe you will be beheaded, if they are mercifull. You might as well be burned on the stake or, hung or, at worst, totured for a long time and lastly impaled and presented as a warning for anyone else to not fight them. Needless to say, you will loose your kingdom.

\subsection{Downtime}

If you, for any reason choose to take down time to administrate your Head Quarter, you may do what you like. First of all, specify how long you want to Administrate the Headquarter and what you want to achieve. You may take over the role of only one of your Personal by doing so. Your GM will then ask you for some roles and/or stats in general and determin the outcome of your downtime.

Everything can happen within that downtime. Your constalation determins the likely hood of certain events. If you, for example, let an Warlord (or an player-equivalent) make the job of an Nobel, war might break loose. Or, if you choose to become a Builder, even tho you clearly are a magician, who lived in the wood for the last 100 years, mulitple buildings might collapse.

In gerneal, those possibilities are low. But they exist. Be as clear as possible about what you want to do, achive and how you want to treat certain people in your specified downtime. You GM might than ask you some questions like "Where do you get the food for the worker from" and determin the outcome. At this downtime, you might get a higher stat, or become proficient in something. But, since this downtime is not determined to train you in certain skill, but to maintain and adminstrate the Headquarter, this is a rare scenario.
