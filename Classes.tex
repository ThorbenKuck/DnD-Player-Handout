\chapter{Chapter 4: Classes}
\twocolumn[{%
\begin{@twocolumnfalse}
\header{The Necromancer Table}
\begin{dndtable}[ l l p{13 cm} ]
\textbf{Level} & \textbf{Proficiency Bonus} & \textbf{Feature} \\
1st & +2 & Controll Undead, Raise Undead\\
2st & +2 & Lifetap\\
3st & +2 & Necromantic Aspiration\\
4st & +2 & Ability Score Improvement\\
5st & +3 & Charnel Touch, Aura of Undead\\
6st & +3 & Necromantic Aspiration feature\\
7st & +3 & Dark Nova\\
8st & +3 & Ability Score Improvement\\
9st & +4 & Control and Raise Undead Improvement\\
10st & +4 & Sacrifice Undead, Necromantic Aspiration Feature\\
11st & +4 & Aura of Undeath Improvement, Dark Nova improvement, Charnel Touch improvement\\
12st & +4 & Ability Score Improvement\\
13st & +5 & Dominate Undead\\
14st & +5 & Necromantic Aspiration feature\\
15st & +5 & Ethereal Mind\\
16st & +5 & Ability Score Improvement\\
17st & +6 & Aura of Undeath Improvement, Dark Nova improvement, Charnel Touch improvement\\
18st & +6 & Raise Undead Improvement\\
19st & +6 & Ability Score Improvement\\
20st & +6 & Death Gerneral
\end{dndtable}
\end{@twocolumnfalse}
}]

\section{Necromancer}
While others use magic to do paltry things like conjure fire or fly, the Necromancer is a master over death itself. They study the deep and forbidden secrets that raise the dead, controlling minions toward a variety of goals. Perhaps they seek the power that mastery over death provides. Perhaps they are serious and unashamed scholars, who reject the small-minded boundaries held to by others. Each enemy they fell becomes an eager and disposable ally, they become immune to the energies of death and decay, and ultimately harness the immortality and power of undeath for themselves.

\subsection{Undead Minions}
Necromancers exert control over undead creatures as if they were part of the same creature. Unlike hordes of wild undead, the undead controlled by a necromancer act as one unit, often running (or shambling) to each other’s aid when a member of the horde is injured or destroyed.

\newpage
\subsection{A Life of Consequences}
Necromancers don’t have the best history with people. Though the large majority of necromancers merely want to help people, whether it be in communicating with dead family members or overthrowing tyrannical kings, evil necromancers are by far more well known for the occasional invasion of a peaceful city, which people don’t tend to take kindly to.

A necromancer must always be careful of his company, because while many people will seek to understand his intentions, just as many will ignore them altogether in the assumption he is evil.

\subsection{Creating A Necromancer}
A Necromancer is a caster that is able to expel negative energies flowing through their veins. Necromancers are similar to sorcerers, but are more adept with necromancy and, to some extent, enchantment spells. They use their abilities to gain absolute control over their enemies' bodies, minds and souls. Often the best way to do this is by raising/summoning undead from their fallen enemies; a skill at which they are unparalleled. Necromancers are also effective with diseases, poison spells, and affecting opponents with fear, fatigue, exhaustion, pain, negative energy damage, or even gaining mindless supporters through the use of enchantment spells like charm or dominate.

A Necromancer's strengths are in bolstering undead, summoning or raising undead minions (which they can control up to a number of a large mob) and being able to cast a vast repertoire of various necromancy spells. They are strong spell casters but are not durable in physical combat. A Necromancer should primarily be used for crowd control, able to curse the enemy while animating different undead to occupy the enemy while their teammates continue to sustain a mass of dead bodies for you.

The most important thing to consider when creating a necromancer are the reasons you became a necromancer. Were you driven to the edge when someone you love was killed or did an oppressive government lead to your seeking rebellion? Either way, a necromancer is often misunderstood by those around them. Necromancers must usually hide their abilities to avoid persecution.

\subsubsection{Quick Build}
You can make a warlock quickly by following these quggestions. First, Charisma should be your highest ability score, followed by Constitution. Second, choose the Sage background.

\subsection{Class Features}
As a Necromancer, you gain the following class features.

\paragraph{HIT POINTS}\mbox{}\\
\textbf{Hit Dice:} 1d6 per necromancer level\\
\textbf{Hit Points at 1st Level:} 6 + your Constitution modifier\\
\textbf{Hit Points at Higher Levels:} 1d6 (or 4) + your  Constitution modifer per necromancer level after 1st

\paragraph{PROFICIENCIES}\mbox{}\\
\textbf{Armor:} Light armor\\
\textbf{Weapons:} Simple weapons, scythes\\
\textbf{Tools:} Embalming Tools\\
\textbf{Saving Throws:} Charisma, Constitution\\
\textbf{Skills:} Choose 2 from: Arcana, Deception, History, Insight, Intimidation, Medicine, Persuasion, and Religion

\subsubsection{Equipmet}
You start with the following equipment, in addition to the equipment granted by your background:
\begin{itemize}
\item (a) A scythe or (b) 2 daggers
\item (a) Leather armor or (b) padded armor
\item (a) A scholar's pack or (b) a explorer's pack
\item An arcane focus (spellcasting focus)
\item An embalming tools
\end{itemize}

\subsection{Control Undead}
At 1st level, you gain the ability to bring wild undead under your control by force. As an action, target an uncontrolled undead with an Intelligence lower than 8 within 30 feet of you. The undead makes a Charisma save DC 8 + your proficiency bonus + your Charisma modifier, and if it fails it is brought under your control indefinitely, but if it succeeds you cannot use this feature on it again for 24 hours. You may only control undead in this way that do not have a CR greater than your level. When you take over undead in this way, they are considered animated by you, and if you do not already control the undead you take damage equal to 1 per 1/8 CR of undead controlled (minimum 1) and your HP maximum is reduced by an equal amount. This Damage and HP reduction cannot be reduced by any means. Your hit point maximum is restored as the controlled undead die or leave your control, but you are not healed. You may control a maximum number of Undead equal to your Charisma modifier (minimum 1) times your Necromancer level.

You may control any undead you control as an action on your turn. At 9th level, this changes to a bonus action. If you are not actively controlling your undead, they will attempt to execute your last orders to the best of their ability. You only maintain control of your undead as long as they are within 1 mile of you. If they leave a 1 mile radius, they leave your control and act as normal undead.
Undead you control report to you psychically any creatures or environment that they can see. You innately know the general direction and distance of all controlled undead.

\subsection{Raise Undead}
Starting at 1st level, you may use your own life force to animate recently dead corpses. Using your action, you may touch a creature that died in the last minute and raise it as an undead under your control indefinitely.  You must spend an hour raising a creature if it has been dead for longer than a minute. Whenever you use this class feature, you take damage equal to 1 per 1/8 CR of undead controlled (minimum 1) and your HP maximum is reduced by an equal amount. This Damage and HP reduction cannot be reduced by any means. Your hit point maximum is restored as the controlled undead die or leave your control, but you are not healed. You may only raise undead in this way that do not have a CR greater than your level, and you are limited to undead with an intelligence lower that 8. At 9th level, you gain the ability to create undead with an intelligence score of 8 or higher, but it must be less than 13. At 18th level, you no longer have restrictions on the kinds of undead you can create based on intelligence score.

\subsection{Lifetap}
At 2nd level, you learn to freely manipulate your own life energy. As an action, you may touch any undead you have animated and deal any amount of damage to it up to it’s current health, giving you temporary hit points or healing any other undead you can touch that you have animated for an amount equal to the damage dealt. Alternatively, you can use an action to touch any creature and channel your own life force into it. This damages you for any amount up to half your maximum hit points or your current hit points, whichever is lower, that you choose and then heals the target for an equal amount.

\subsection{Necromantic Aspiration}
At 3rd level choose a Necromantic Aspiration: Caretaker, Reaper, and Summoner, all detailed at the end of the class description. Your choice grants you features at 3rd level, and again at 6th, 10th, and 14th level.

\subsection{Ability Score Improvement}
When you reach 4th level, and again at 8th, 12th, 16th, and 19th level, you can increase one ability score of your choice by 2, or you can increase two ability scores of your choice by 1. As normal, you can’t increase an ability score above 20 using this feature.

\subsection{Charnel Touch}
At 5th level, you learn to touch the lives of your foes and steal them for yourself. As an action, you may make a ranged spell attack (proficiency + Charisma modifier) against a living creature within 60 feet. If the attack hits, the target takes 1d8 damage plus your charisma modifier. This damage increases to 2d8 at 11th level, and to 3d8 at 17th level.
Additionally, if the target of the attack dies, you gain temporary hit points equal to half the damage dealt.

\subsection{Aura of Undeath}
Starting at 5th level, you gain an aura of undeath. You may use a bonus action to activate any or all auras you know and you may turn any of them off at any time for free on your turn.
You choose one of the following auras when you gain this feature, and again at 11th and 17th levels. When your Aura of Undeath is turned on you maintain any auras you choose until you dismiss the effect. You may use a bonus action on future turns to activate additional Aura effects, but may dismiss them for free. Your Aura of Undeath has a radius of 30 ft, and affects all undead you control within it. Some Aura effects require you to take damage at the beginning of your turns to maintain them. This cost is noted on each Aura effect.

If an Unholy Aura and a Holy Aura overlap, creatures in the overlapping areas get none of the benefits of either (so Aura of Undeath and a Paladin’s Devotion Aura would cancel out).\\

\textbf{Aura of Ferocity}\\
Costs 2 hp/round.\\
Affected creatures may add your Charisma modifier to their damage rolls with weapon attacks.\\

\textbf{Aura of Resilience}\\
Costs 1 hp/round.\\
Affected creatures may add your Charisma modifier to any saving throws they make.\\

\textbf{Aura of Retaliation}\\
Costs 2 hp/round.\\
Affected creatures may make an attack of opportunity against any creature that attacks them with a melee weapon or melee spell attack.\\

\textbf{Aura of Tenacity}\\
Costs 3 hp/round.\\
Affected creatures take less damage from non magical bludgeoning, piercing, and slashing damage equal to half your Charisma Modifier, rounded up (minimum 1).\\

\textbf{Aura of Terror}\\
Costs 2 hp/round.\\
Affected creatures become more menacing. Any enemy creature that starts its turn or moves within 5 feet of an affected creature must make a wisdom saving throw against DC 8 + your proficiency bonus + your charisma modifier or become frightened for 1 minute. They may repeat this save at the end of each of their turns. If they succeed on their save, they become immune to this effect for 24 hours. Any creatures that are frightened when this aura ends stop being frightened.

\subsection{Dark Nova}
At 7th level, you gain the ability to channel your life force to damage nearby enemies. As an action, you release Dark energy in a 10 ft. radius sphere with yourself as the point of origin. Creatures hit by this burst of dark energy may make a Constitution saving throw DC = 8 + your proficiency bonus + your Charisma modifier. 

On a failure they take 3d8 necrotic damage and are pushed 5 feet away from you. On a successful save, they take half damage and are not pushed. This damage increases to 5d8 at 11th level, and to 7d8 at 17th level. You may use this ability a number of times per long rest equal to your Charisma Modifier.

\subsection{Sacrifice Undead}
Starting at 10th level, you gain the ability to sacrifice your controlled undead to restore your life when you would be knocked unconscious. As a reaction to taking damage that would reduce you to 0 hit points, you may sacrifice an undead you control within 30 feet to instead drop to 1 hit point. You may use this feature once per long rest.

\subsection{Dominate Undead}
At 13th level, the abilities of your Control undead feature extend to intelligent undead and undead controlled by other necromancers. Intelligent undead are harder to control in this way. If the target has an Intelligence of 8 or higher, it has advantage on the saving throw. If it fails the saving throw and has an Intelligence of 12 or higher, it can repeat the saving throw at the end of every hour until it succeeds and breaks free.
If an undead you are trying to control is controlled by someone else, you may instead use your action to initiate a contest against said creature. Both of you roll 1d20 and add your Proficiency bonus and Charisma modifiers. If you win the contest, the undead is brought under your control, but in the event of a tie or if you lose, nothing happens. Once an undead has been contested like this, it cannot be contested again for 1 hour. If the undead has an Intelligence of 8 or higher, it may grant advantage in this contest to either necromancer.

\subsection{Ethereal Mind}
Starting at 15th level, your knowledge of necromancy allows you to understand concepts of spirituality foreign to normal people. First, once per day, you may use your action to see 60 feet into the ethereal plane for 10 minutes.
Additionally, once per day, when touching a dead body, you may begin a 1 hour ritual, during which you may converse freely with the soul that previously inhabited it, provided the soul is willing. If you have a possession of the spirit you are trying to contact, you may also use that to contact them

\subsection{Death General}
At 20th level, you gain the ability to choose a single controlled undead as your Death General. This undead can be chosen from any undead you control and gains a number of additional benefits.
Creating a Death General requires performing an 8 hour ritual every day for a week and 5,000 gp worth of materials.
\begin{itemize}
\item Your General’s HP cost is increased by 20 HP.
\item Your General’s Hit Points are their normal HP or 100, whichever is higher.
\item Your General’s Intelligence, Wisdom, and Charisma are replaced with your own, and they add your weapon and armor proficiencies to their own.
\item Your general has a control radius equal to your own and you may control undead within that radius. Your General can be controlled as long as you exist on the same plane of existence, and he will act to reach you if you are separated in such a way.
\end{itemize}
When you manifest your Aura of Undeath, Control Undead, or Dark Nova class features, you may do so from your General’s location, but they do not manifest from your own location if you do this.
You may use an action to begin seeing through your General’s senses. This lasts until you end it and causes you to become blind and deaf to anything around your own body.
You innately know the general direction and distance of any undead within the General’s control radius, and if an undead is within both yours and your Generals radius you know it’s exact location.
Any time you would die for any reason. Your General dies instead, and your body is teleported to a safe place chosen by your DM on the same plane of existence, unconscious but stable. Any undead you control leave your influence and become wild.\\

Alternatively, you may choose to become the Death General yourself. Becoming a Death General requires performing an 8 hour ritual every day for a week, but has no gold cost. When you do this, you gain the following features.

\begin{itemize}
\item Your Maximum HP is increased by 40, but these hit points can not be used to control undead.
\item You gain proficiency in Wisdom saving throws.
\item Your Charnel Touch deals an additional 1d8 damage.
\item The ranges of your Dark Nova and Aura of Undeath are doubled.
\item Your undead control radius is doubled.
\end{itemize}

\subsection{Necromantic Aspirations}
Necromancers share their affinity with undead, but how they treat their undead often varies. A Necromancer that minds their undead carefully is very different from a Necromancer that sits back while his army wages war, and both of these are very different from a Necromancer that fights on the front line with their undead as their leader.

\subsection{Caretaker}
Care for and enhance a small group of undead.

\subsubsection{Enhanced Animation}
At 3rd level, the damage and max HP cost per CR of animating undead increases by your proficiency bonus. As a result, undead you control may add your proficiency bonus to hit and have bonus HP equal to your necromancer level.

\subsubsection{Enhanced Resilience}
Starting at 6th level, undead you create have AC = 8 + your proficiency bonus + their dexterity modifier.

\subsubsection{Conscious Animations}
Starting at 10th level, your animations may maintain a facet of their living consciousness. These undead are capable of making their own decision, and have an intelligence and wisdom no lower than 10.

\subsubsection{Dark Restoration}
Starting at 10th level, your Dark Nova feature heals undead you control for an amount equal to half the damage they would have taken.

\subsubsection{Selective Binding}
Starting at 14th level, you gain the ability to animate bodies by binding willing spirits to them. You must spend an hour binding the spirit to the body, allowing the spirit to animate it. The ghost maintains the ability to leave the body, ending this effect. The animated body acts as if it were a normal undead, but is controlled by the spirit.


\subsection{Reaper}
Cut down your enemies with your own unholy powers.

\subsubsection{Bonus Proficiencies}
At 3rd level, you gain proficiency in Medium Armor and Martial Weapons.

\subsubsection{Dark Strike}
Starting at 3rd level, your melee weapon attacks deal bonus necrotic damage equal to your charisma modifier (minimum 1) once per turn. When you deal damage with this feature, you are healed for an amount equal to half your bonus damage rounded up (minimum 1).
At 10th level, your Dark Strike deals an additional 1d8 necrotic damage. This increases to 2d8 at 14th level.

\subsubsection{Extra Attack}
Starting at 6th level, you may make an additional attack when you take the attack action.
\subsubsection{Warcaster}
At 10th level you gain the ability to interweave your weapon attacks and your magic. When you take the Attack action on your turn, you may cast Charnel Touch or Dark Nova as a bonus action.
\subsubsection{Soul Reaper}
Starting at 14th level, whenever you kill a creature with Dark Strike, you may use your Raise Undead Class Feature on the creature you killed as a bonus action.


\subsection{Summoner}
Summon massive hordes of undead to crush your enemies.

\subsubsection{Efficient Animation}
At 3rd level, the damage and max HP cost per CR of animating undead decreases by your proficiency bonus when creating or controlling undead of a CR less than or equal to half your level (rounded up) or lower.

\subsubsection{Summon Undead}
Starting at 6th level, you no longer require a corpse to create undead, as you can just as easily summon them to a space within 5 feet of you directly from another plane such at the Shadowfell or the Negative Energy plane. This process takes 1 hour.

\subsubsection{Grandmaster}
Starting at 10th level, you gain the ability to control your undead from up to 10 miles away. This increases to 20 miles at 14th level.

\subsubsection{Unholy Siegemaster}
At 14th level, you gain the ability to create undead siege engines. These siege creatures function exactly as if they were normal siege equipment, but are considered controlled undead with a CR of 1 per 10 HP they have (so a Mangonel with 100 hp would be considered CR 10, or 75 hp is CR 7). Each siege engine takes 1 hour per CR to create and does not require a crew to operate.

\newpage\cleardoublepage

% ##########################################################################################
% # CHRONOMANCER, THE TIMEKEEPER                                                           #
% # source: https://www.dandwiki.com/wiki/Chronomancer,_Guardian_and_Timekeeper_(5e_Class) #
% ##########################################################################################

\section{Chronomancer}

\subsection{Time and Fate's Chosen}

Chronomancers are often the stuff of legend this is because they have been around since the start of time and new chronomancers are only ever created once a previous one has requested Time and Fate to put them in an eternal slumber. New chronomancers are mortals that have caught Time and Fate's eye either by studying time manipulation or being gifted with extraordinary speed to the extent that the world moves slower to them. Once selected the chronomancer is infused with astral energy which flows through everything and learns how to tap into it and utilise the timeline to their advantage. Once infused, astral energy then flows over the chronomancer's body acting like a focus of astral energy.

\subsection{Maintaining the Timeline}

There are those who travel to the furthest reaches of the Astral Plane and return telling of stories of how they found a temple-like structure being guarded by creatures with blue lightning flowing over their skin. Though often not studied too far, chronomancers are essential for the creatures of the planes, each chronomancer sets down a path and promises to Time and Fate that they will use all their power to maintain it, eliminating threats to its existence for the betterment of human kind. Once selected by the deities the new chronomancer is placed in the Material Plane and left there to guide their timeline with small actions, as they train they gain more favour by Time and Fate until they reach the point where they are summoned to the Astral Sanctum, the home of Time and Fate, where they assist the gods directly.

Some creatures that study time and the chronomancers wish to be taken into the arms of the two gods, trying to draw their attentions in both constructive and destructive ways. Others are just removed from their normal lives by the gods somewhat randomly.


\subsection{Multiclassing}
\textbf{Prerequisites} To qualify for multiclassing into the Chronomancer class, you must meet these prerequisites: 15 Intelligence and a Teacher, who in fact is a Chronomancer\\
\textbf{Proficiencies}  When you multiclass into the class, you gain the following proficiencies: none\\

\twocolumn[{%
\begin{@twocolumnfalse}
\header{The Guardian Table}
\begin{dndtable}[ l l p{11 cm} l ]
\textbf{Level} & \textbf{Proficiency Bonus} & \textbf{Feature} & \textbf{Vortex Damage} \\
1st & +2 & Unarmored Defense, Astral Equipment	& -\\
2st & +2 & Vortex Strikes, Rift Heal & 1d4\\
3st & +2 & Timeline & 1d4\\
4st & +2 & Ability Score Improvement & 1d4\\
5st & +3 & Extra Attack, Vortex Strikes Improvement & 1d6\\
6st & +3 & Timeline Feature & 1d6\\
7st & +3 & - & 1d6\\
8st & +3 & Ability Score Improvement & 1d6\\
9st & +4 & Time Stride & 1d6\\
10st & +4 &	Extra Attack (2) & 1d6\\
11st & +4 &	Timeline Feature & 1d8\\
12st & +4 &	Ability Score Improvement & 1d8\\
13st & +5 &	Languages of Time & 1d8\\
14st & +5 &	Timeline of War & 1d8\\
15st & +5 &	Timeless Body & 1d8\\
16st & +5 & Ability Score Improvement & 1d8\\
17st & +6 &	Timeline Feature & 1d10\\
18st & +6 &	Chrono Plating & 1d10\\
19st & +6 & Ability Score Improvement & 1d10\\
20st & +6 & Rupture & 1d10
\end{dndtable}
\end{@twocolumnfalse}
}]

\section{Chronomancer, The Guardian}

Letting out a howl, the stout dwarf draws the attention of many of the orcs each one uninterested in the other fighters around him. They bring blade after blade down on him but with unnatural speed the dwarf moves his shield, rapidly deflecting each attack. He was going to keep his allies alive, that was what he was here for.

The human starts to walk calmly out towards the incoming hoard of hobgoblins, this city must stand if the timeline is to remain. Raising one arm out to his side dark-blue lightning sparks from the surrounding air, it solidifies to form a long glowing blue blade ready in his hands. He already knows the outcome of this battle, he smiles and charges towards his victory.

The elf laughs as he tears his weapon through the fabric of space and time creating an open gateway to the Vortex of Time itself. He stares into it as an energy flows out and into his body. Without chaos how can order exist, he thinks to himself as he lets the vortex flow and surround his body. Those who looked upon him for too long fell to the ground muttering madly.

\newpage

Guardians are brute enforcers of the Time and Fate deities tasked with maintaining the timeline. They focus on being able to deal damage and removing any threat to their timelines existence. They let their control of time aid them and hinder their enemies. Although, meddling with time is a tough ordeal hence chronomancers are very rare.

\subsection{Creating a Guardian}

As you create a chronomancer, think about what got you noticed by Time and Fate. Were you one of the far walkers who discovered the Astral Sanctum? Were you someone who always felt that time was on their side for some reason? Did you find that the world felt like it moved slower around you?

Did you accept the roll of the chronomancer happily? Or did you purposely try to grab the attention of the gods to be selected as a chronomancer? Or were you rather forcefully ripped from your normal life? Do you wish to return to your previous life?

\subsubsection{Quick Build}

You can make a guardian quickly by following these suggestions. First, Strength should be your highest ability score or Constitution if you plan to take the Righteous Timeline archetype, followed by Intelligence. Second, choose the soldier background.

\subsection{Class Features}

\paragraph{HIT POINTS}\mbox{}\\
\textbf{Hit Dice:} 1d10 per Chronomancer level\\
\textbf{Hit Points at 1st Level:} 10 + your Constitution modifier\\
\textbf{Hit Points at Higher Levels:} 1d10 (or 6) + your  Constitution modifer per Chronomancer level after 1st

\paragraph{PROFICIENCIES}\mbox{}\\
\textbf{Armor:} Light armour, medium armour\\
\textbf{Weapons:}Martial weapons, shields\\
\textbf{Tools:} None\\
\textbf{Saving Throws:} Strength, Intelligence\\
\textbf{Skills:} choose any 2 from Arcana, Athletics, History, Insight, Medicine, Nature, Religion, Survival

\paragraph{Equiment}

You start with the following equipment, in addition to the equipment granted by your background:
\begin{itemize}
\item (a) a martial weapon and a shield or (b) two martial weapons
\item a longbow and 20 arrows
\item (a) a dungeoneer's pack or (b) an explorer's pack
\end{itemize}

\subsection{Unarmored Defense}

Beginning at 1st level, while you are not wearing any armour, your AC equals 10 + your Dexterity modifier + your Intelligence modifier. You can use a shield and still gain the benefit.

\subsection{Astral Equipment}

At 1st level, you have trained the ability to tap into the astral energy that surrounds all life in all planes of existence and can mould it to your will. Taking a bonus action, you can summon, or change a weapon or shield in either hand. You can summon two weapons as long as they have the light property for instance, you can summon two shortswords or one shortsword and one shield. If you are playing the optional feat rules from chapter 5 in the Player's Handbook and take the Duel-Wielding feat you can summon two weapons even if they don't have the light property as per the feat. You are proficient with any weapon you summon. 

Beginning at 1st level you can summon;
\begin{itemize}
\item Longsword. 1d8 slashing. Versatile (1d10).
\item Shortsword. 1d6 piercing. Finesse, light.
\item Longbow. 1d8 piercing. Ammunition (range 150/600), heavy, two-handed. This weapon is summoned with 15 Arrows.
\item Shield. +2 AC.
\end{itemize}
At 4th level you can summon;
\begin{itemize}
\item Flail. 1d8 bludgeoning.
\item Rapier. 1d8 piercing. Finesse.
\end{itemize}
At 8th level you can summon;
\begin{itemize}
\item Greataxe. 1d12 slashing. Heavy, two-handed.
\item Maul. 2d6 bludgeoning. Heavy, two-handed.
\end{itemize}
At 12th level you can summon;
\begin{itemize}
\item Greatsword. 2d6 slashing. Heavy, two-handed.
\item Morningstar. 1d8 piercing.
\end{itemize}
At 16th level, you can summon any martial weapon.

The summoned weapon lasts for 10 minutes and will automatically dispel after this time.

\subsection{Vortex Strikes}

Every weapon strike you make ripples with chaotic energy from the time vortex pulling and twisting everything it hits. Beginning at 2nd level, every weapon you summon has the vortex property. Every time you hit a creature with a weapon with the vortex property you gain a bonus to the damage roll that increases as you gain levels as a guardian, as shown in the Vortex Damage column of the Guardian table. This damage is psychic for the purposes of immunities and resistances.

At 5th level, attacks by weapons that have the vortex property count as magical for the purpose of overcoming resistance and immunity to nonmagical attacks and damage.

\subsection{Rift Heal}

Beginning at 3rd level, you learn how to channel the astral energy of others into your own life essence. When you deal additional vortex damage to a creature you gain hitpoints equal to your level, this cannot exceed your maximum hit points.

Once you use this feature, you can't use it again until you finish a short or long rest

\subsection{Timeline}

When you reach 3rd level, you dedicate yourself to maintaining a Timeline: The Righteous Timeline, the True Timeline, or the Disordered Timeline, all detailed at the end of the class description. Your Timeline grants you features at 3rd, 6th, 11th, and 17th level.

\subsection{Ability Score Improvement}

When you reach 4th level, and again at 8th, 12th, 16th, and 19th level, you can increase one ability score of your choice by 2, or you can increase two ability scores of your choice by 1. As normal, you can't increase an ability score above 20 using the feature.

\subsection{Extra Attack}

Beginning at 5th level, you can attack twice, instead of once, whenever you take the Attack action on your turn.

The number of attacks increases to three when you reach 10th level in this class.

\subsection{Time Stride}

Starting at 9th level, your understanding of how time flows allows you to move at extraordinary speed, seemingly teleporting a short distance. You must make an Intelligence saving throw DC 10, on a success you can teleport 30 feet as a bonus action, on a failed save you take 2d12 psychic damage. For each consecutive use increase the DC by 5 to a maximum of DC 20.

The check DC returns to 10 after you finish a short or long rest.

\subsection{Languages of Time}

Starting at 13th level, you can peer into the timeline and see how languages have come to be allowing you to understand all spoken languages. Moreover, any creature that can understand a language can understand what you say.

\subsection{Timeline of War}

Beginning at 14th level, having seen the worst that battle has brought to this realm since the beginning of time you can no longer be fazed by what war can bring. You can take an action to end one effect on yourself that is causing you to be charmed, frightened, paralyzed, poisoned, or stunned.

\subsection{Timeless Body}

At 15th level, your body no longer remains to this time stream, you no longer age and cannot be aged magically. In addition, you no longer need food or water.

\subsection{Chrono Plating}

You channel the astral energy from the environment around you providing a solidified layer of pure energy over your skin. Beginning at 18th level, you gain resistance to psychic damage; bludgeoning, piercing, and slashing damage from nonmagial weapons.

\subsection{Rupture}

Beginning at 20th level, you can create ruptures in the continuum of time. You have a pool of 6 rift marks when you hit a creature with vortex weapon you may use a bonus action to mark the creature once (reducing your pool by one). You can mark the same creature multiple times. You can take an action to damage all creatures that have been marked, all creatures that have been marked take vortex dice (as shown in the Vortex Damage column of the class table) + 1d12 for each mark on the creature.

You regain all expended rift marks when you finish a long rest.

\subsection{Timelines}

\subsection{The Righteous Timeline}

Chronomancers who walk the Righteous Timeline strive to maintain the most positive timeline, they wish to create the best timeline for all living creatures. They focus on techniques allowing them to improve the way of life for their allies, using the astral energy to heal and guide enemies away from an ally happily taking the damage in order to protect others. Chronomancers who follow this timeline are usually of the good alignment.

\subsubsection{Saviour}

When you start your path down the Righteous Timeline at 3rd level, the lives of those around you seem far more important than your own and you'll take damage for them.

As an action, you can connect yourself to your allies with a beam of dark-blue energy reducing some of the damage they take for 1 minute. When you do so, choose a number of allies you can see within 30 feet of you, up to a number of them equal to your Intelligence modifier (minimum of 1). As a reaction, when one of the targets takes damage, you may take damage equal to 10 + your Guardian level and reduce the damage the target takes by the same amount. If a target ends its turn more than 30 feet from you or is no longer in your line of sight the effect ends on that creature.

Once you use this feature, you can't use it again until you finish a short or long rest. You can use this feature two times per rest at 7th level and three times at 14th level.

\subsubsection{Compelling Presence}

At 6th level, you use your control over the astral energy and the timeline to draw the attention of any creature that can see you to yourself. As a bonus action, choose a number of creatures that you can see and who can see you within 30 feet of you, up to a number of them equal to your Intelligence modifier (minimum of 1). Each target must succeed on a Wisdom saving throw (DC equals 8 + your proficiency modifier + your Intelligence modifier) or be compelled to attack you for 1 minute.

A creature affected by this feature can't willingly move away from you and has disadvantage on attack rolls against any creature other than you for the duration. A creature can repeat the saving throw at the end of each of its turns, ending the effect on itself on a success. If a creature's saving throw is successful or the effect ends for it, the creature is immune to Compelling Presence for the next 24 hours.

Once you use this feature, you can't use it again until you finish a short or long rest.

\subsubsection{Astral Channelling}

Beginning at 11th level, you can convert the surrounding astral energy into healing energy and give it to your allies. As an action, you can choose any number of allies you can see and who can see you within 30 feet of you. Each target gains hit points equal to your vortex damage dice as shown in the Vortex Damage column in the Guardian class table. You can use this feature times equal to your Intelligence modifier.

You gain all expended uses when you finish a short or long rest.

\subsubsection{Defender of Righteousness}

At 17th level, you can tell when a creature is about to attack an ally and manipulate the timeline so you are attacked instead. When an ally you can see within 30 feet of you is targeted by an attack you can use your reaction to teleport to the attacking creature. The attacking creature can must switch attack you instead or lose the attack. You can use this feature times equal to your Intelligence modifier.

You gain all expended uses after you finish a short or long rest.

\subsection{The True Timeline}

Those who walk the True Timeline know that in order for life to continue advancing there must be balance in life and death, good and evil. The chronomancers who follow this timeline aim to maintain balance, they focus on being offensive aggressors aiming to remove those who risk disrupting the timeline. These chronomancers are often true neutral alignment, for every wrong there is a right and vice versa.

\subsubsection{Fighting Style}

You adopt a particular style of fighting as your speciality when you select this timeline at 3rd level. Choose one of the following options.

\textbf{Archery.} You gain a +2 bonus to attack rolls you make with ranged weapons.\\
\textbf{Dueling.} When you are wielding a melee weapon in one hand and no other weapons, you gain a +2 bonus to damage rolls with that weapon.\\
\textbf{Great Weapon Fighting.} When you roll a 1 or 2 on a damage die for an attack you make with a melee weapon that you are wielding with two hands, you can reroll the dice and must use the new roll, even if the new roll is a 1 or a 2. The weapon must have the two-handed or versatile property for you to gain this benefit.\\
\textbf{Two-Weapon Fighting.} When you engage in two-weapon fighting, you can add your ability modifier to the damage of the second attack.\\

\subsubsection{Marked Through Time}

Starting at 6th, those who don't belong on, or bring risk to the True Timeline give off and aura that you can focus on. As a bonus action, you can mark a creature you can see within 30 feet of you. You gain advantage on attack rolls against the creature and you know the distance and direction between you and the marked creature, if it is on the same plane of existence as you, for 10 minutes or until it drops to 0 hit points or falls unconscious.

Once you use this feature, you can't use it again until you complete a short or long rest.

\subsubsection{Astral Shell}

Beginning at 11th level, you surround yourself with unstable astral energy which could be unleashed if any damage is done to the shell. As a reaction, which you take when you are hit by an attack, you can cause the attacking creature to take vortex damage shown in the Vortex Damage column of the Guardian table.

You can use this feature a number of times equal to half your Guardian level rounded down. You regain all expended uses once you finish a long rest.

\subsubsection{Astral Pulse}

At 17th level, you are able to channel the raw energy of the astral plane, using it to destroy all those who have meddled with the True Timeline. As an action, hostile creatures within 30 feet of you must make a Constitution saving throw. A creature takes psychic damage equal to 2d10 + your Guardian level on a failed saving throw, and half as much on a successful one. A creature that has total cover from you is not affected.

You can use this feature twice per long rest.

\subsection{The Disordered Timeline}

Chronomancers who tread the Disordered Timeline respect the chaotic nature of the time vortex letting this chaotic existence guide them. They tap into the vortex itself instead of learning how to master astral energy, this allows them to access the near unlimited power of the vortex however, the outcome is as random as the vortex itself. Chronomancers who travel down this timeline are often of chaotic alignment.

\subsubsection{Time Chaotic Gift}

When you select this timeline at 3rd level you gain the ability to draw power out of the vortex of time. As an action, you cause a small rift into the vortex and accept whatever power it grants you when you do roll a d4 and a d6;

\begin{dndtable}[ccX]
\textbf{d4} & \textbf{d6} & \textbf{Effect}\\
1 & 1-5 & \textbf{Rift Shield.} A chaotic layer of energy covers your skin, all attacks against you have disadvantage for 1 minute.\\
 & 6 & \textbf{Ripped Soul.} The chaotic vortex partially divides your mind from your body causing you to move less defensively, all attacks against you have advantage for 1 minute.\\
2 & 1-5 & \textbf{Chaos Strike.} Energy from the vortex flows over your weapon imbuing it with greater power, for 1 minute you gain an increase to hit and damage rolls with your current weapon, this increase is equal to your Intelligence bonus.\\
 & 6 & \textbf{Draining Strike.} The vortex of time weathers your weapon through the ages, its blade getting blunt. For 1 minute, you have a penalty to hit and damage rolls with your weapon, this penalty is equal to your Intelligence modifier. For instance, a creature with Intelligence 14 (+2) would have a penalty of -2 to hit and to damage.\\
3 & 1-5 & \textbf{Vitality.} You feel the vortex reverse your wounds. You gain hit points equal to twice your guardian level.\\
 & 6 & \textbf{Lethargy.} The chaotic void pulls your life essence out of you. You take psychic damage equal to twice your guardian level.\\
4 & 1-5 & \textbf{Swift.} The vortex surrounds your local time bubble speeding it up. you gain the effect of a haste spell that lasts for 1 minute. Additionally, when you make a running jump your jump distance increases by a number of feet equal to your Strength modifier.\\
 & 6 & \textbf{Sluggish.} The speed of your local time bubble slows down to a snail's pace. Your speed is halved, you take a -2 penalty to AC and Dexterity saving throws, and can't use reactions for 1 minute. Additionally, you can only use an action or a bonus action, not both.\\

\end{dndtable}

\subsubsection{Timeline of Chaos}

Beginning at 6th level, you spend your time peering into the dark parts of the Timeline knowing what others fear. Taking an action, a target that can see you must make a Wisdom saving throw, on a failed save the target is frightened for 1 minute. Additionally, you gain proficiency in the Intimidation skill, if you are already proficient you can double your proficiency when rolling for Intimidation checks.

\subsubsection{Aura of Altered Time}

Starting at 11th level, once per long rest you can take an action to create a 10-foot radius aura around you, roll a d6 to find its effect;

\begin{dndtable}[cX]
\textbf{d6} & \textbf{Effect}\\
1-2 & \textbf{Aura of Guidance.} You use your ability to see your enemies’ weaknesses, you can guide your allies' weapons to the place where they will do the most damage. Once on their turn, friendly creatures within the radius have advantage on their first attack against a target.\\
3-4 & \textbf{Aura of Dusk.} TYou use your ability to grab your enemies fears and create small illusions only visible to the creature see these images scare and haunt the creature. Any attacks against a friendly target within the radius has disadvantage. If the friendly target is hit by an attack, this effect doesn't start again until your next turn.\\
5 & \textbf{Aura of Credence.} Surfacing memories that make your allies stronger. All friendly creatures within the radius gain advantage on saving throws to resist being frightened, charmed and stunned.\\
6 & \textbf{Aura of the Astral Sea.} You focus on your connection to the Astral Plane and bring a part of it to this plane placing you on limbo of the Astral plane and this plane. All hostile creatures must make an Intelligence save to enter the radius.\\
\end{dndtable}

The save DC for these abilities is equal to 8 + your proficiency bonus + your Intelligence modifier. You must be conscious in order for these effects to take place, the effects of the aura last for 10 minutes. At 14th level, the range of this aura increases to 30 feet.

\subsubsection{Vortex Form}

At 17th level, you form becomes as chaotic as the vortex you gain most of your power from looking into the vortex itself can drive the strongest of men into the depths of insanity. As an action, you allow the vortex to envelop your very being for 10 minutes. Any creature who targets you with an attack or a harmful spell must first make a Wisdom saving throw. On a failed save, the creature must choose a new target or lose the attack or spell. This feature doesn’t protect you from area effects, such as the explosion of a fireball. You can use this feature a number of times equal to your Intelligence modifier.

You regain all expended uses of this feature once you finish a long rest.

\twocolumn[{%
\begin{@twocolumnfalse}
\header{The Timekeeper Table}
\begin{dndtable}[ l l p{3 cm} l l l l l l l l l l ]
\textbf{Level} & \textbf{Proficiency} & \textbf{Feature} & \textbf{Cantrips Known} & \textbf{1st} & \textbf{2nd} & \textbf{3rd} & \textbf{4th} & \textbf{5th} & \textbf{6th} & \textbf{7th} & \textbf{8th} & \textbf{9th} \\
1st & +2 & Spellcasting, Unarmored Defense & 2 & 2 & - & - & - & - & - & - & - & - \\
2st & +2 & Timeline, Psychometry & 2 & 3 & - & - & - & - & - & - & - & - \\
3st & +2 & - & 2 & 4 & 2 & - & - & - & - & - & - & - \\
4st & +2 & Ability Score Improvement & 3 & 4 & 3 & - & - & - & - & - & - & - \\
5st & +3 & - & 3 & 4 & 3 & 2 & - & - & - & - & - & - \\
6st & +3 & Timeline Feature & 3 & 4 & 3 & 3 & - & - & - & - & - & - \\
7st & +3 & - & 3 & 4 & 3 & 3 & 1 & - & - & - & - & - \\
8st & +3 & Ability Score Improvement & 3 & 4 & 3 & 3 & 2 & - & - & - & - & - \\
9st & +4 & - & 3 & 4 & 3 & 3 & 3 & 1 & - & - & - & - \\
10st & +4 & Timeline Feature & 4 & 4 & 3 & 3 & 3 & 2 & - & - & - & - \\
11st & +4 & - & 4 & 4 & 3 & 3 & 3 & 2 & 1 & - & - & - \\
12st & +4 & Ability Score Improvement & 4 & 4 & 3 & 3 & 3 & 2 & 1 & - & - & - \\
13st & +5 & - & 4 & 4 & 3 & 3 & 3 & 2 & 1 & 1 & - & - \\
14st & +5 & Timeline Feature, Language of Time & 4 & 4 & 3 & 3 & 3 & 2 & 1 & 1 & - & - \\
15st & +5 & - & 4 & 4 & 3 & 3 & 3 & 2 & 1 & 1 & 1 & - \\
16st & +5 & Ability Score Improvement & 4 & 4 & 3 & 3 & 3 & 2 & 1 & 1 & 1 & - \\
17st & +6 & - & 4 & 4 & 3 & 3 & 3 & 2 & 1 & 1 & 1 & 1 \\
18st & +6 & Planar Walk, Timeless Body & 4 & 4 & 3 & 3 & 3 & 3 & 1 & 1 & 1 & 1 \\
19st & +6 & Ability Score Improvement & 4 & 4 & 3 & 3 & 3 & 3 & 2 & 1 & 1 & 1 \\
20st & +6 & Astral Infusion & 4 & 4 & 3 & 3 & 3 & 3 & 2 & 2 & 1 & 1 
\end{dndtable}
\end{@twocolumnfalse}
}]

\section{Chronomancer, The Timekeeper}

The human stood behind his allies, carefully observing the battle unfolding in front of him. The undead creature swung its sword in the direction of one of his allies with a quick flick of his hand a barrier of blue astral energy surrounded his friend, the undead's blade sliding harmlessly of the shield. After which he mutters some words and slams his staff into the ground as blue energy flowed from it into his allies. He could see by their expressions that they suddenly felt a wave of energy flow over them as they stood up, the battle now in their favour.

\newpage

The dwarf had found his way to the heart of the enemy standing before him, the mastermind of all the events. He raises his hand and aims it at the Warlord in front of him, and smiles as he reaches into the astral weave and finds his thread and then plucks the thread. The warlord now stands looking at his own figure, no longer of the prime material plane, he looks up at the dwarf fear in his eyes as the dwarf snatches the thread from the true timeline, erasing his existence on this current timestream.

The group of bandits approached the elf weapons drawn getting ready to strike. The elf smiled as she traced her hand in a line straight down in front of her as she ripped open a doorway into the vortex of time. These not very intelligent creatures were unable to draw their gaze from the rift. When the rift faded she remain and look over the bandits, some who met her gaze quickly turned and ran, the others seemed enthralled by her, she uttered some words of command and her new followers quickly obeyed.

Timekeepers are followers of Time and Fate who have been gifted power over the astral energy allowing for the manipulation of time on a greater stage than its guardian counterpart. They focus on the manipulations of energies that surround everything, following their orders to maintain the timeline. They let their control of time aid them and hinder their enemies. Although, meddling with time is a tough ordeal hence chronomancers are very rare.

\subsection{Creating a Timekeeper}
As you create a chronomancer, think about what got you noticed by Time and Fate. Were you one of the far walkers who discovered the Astral Sanctum? Were you someone who always felt that time was on their side for some reason? Did you look too deep into the school of time manipulation? Did you seek Time and Fate?

Did you accept the roll of the chronomancer happily? Or did you purposely try to grab the attention of the gods to be selected as a chronomancer? Or were you rather forcefully ripped from your normal life? Do you wish to return to your previous life?

\subsubsection{Quick Build}

You can make a timekeeper chronomancer quickly by following these suggestions. First, Intelligence should be your highest ability score, followed by Constitution. Second, choose the sage background.

\subsection{Class Features}

\paragraph{HIT POINTS}\mbox{}\\
\textbf{Hit Dice:} 1d6 per Chronomancer level\\
\textbf{Hit Points at 1st Level:} 6 + your Constitution modifier\\
\textbf{Hit Points at Higher Levels:} 1d6 (or 4) + your  Constitution modifer per Chronomancer level after 1st

\paragraph{PROFICIENCIES}\mbox{}\\
\textbf{Armor:} None\\
\textbf{Weapons:} Dagger, darts, slings, quarterstaff, light crossbow\\
\textbf{Tools:} None\\
\textbf{Saving Throws:} Intelligence, Wisdom\\
\textbf{Skills:} choose any 2 from Arcana, History, Insight, Medicine, Nature, and Religion

\subsubsection{Equipmet}
You start with the following equipment, in addition to the equipment granted by your background:
\begin{itemize}
\item (a) A Quarterstaff or (b) a daggers
\item (a) A scholar's pack or (b) a explorer's pack
\item An astral focus 
\item An embalming tools
\end{itemize}

\subsection{Spellcasting}

After focusing on the school of time, or chronomancy, you can cast spells that change the very order of time and can manipulate the astral energies that surround all things in the multiverse to your will. See chapter 10 in the player's handbook for the general rules of spellcasting and see the end of this document for the chronomancer spell list.

\subsubsection{Cantrips}

At 1st level, you know 2 cantrips of our choice from the chronomancer spell list. You learn additional chronomancer cantrips of your choice at higher levels, as shown in the Cantrips Known column of the Timekeeper table.

\subsubsection{Preparing and Casting Spells}

The Timekeeper table shows how many spell slots you have to cast your spells of 1st level and higher. To cast one of these chronomancer spells, you must expend a slot of the spell's level or higher. You regain all expended spell slots when you finish a long rest.

You prepare the list of chronomancer spells that are available for you to cast, choosing from the chronomancer spell list. When you do so, choose a number of chronomancer spells equal to your Intelligence modifier + your timekeeper level (minimum of one spell). The spells must be of a level for which you have spell slots.

For example, if you are a 3rd-level timekeeper, you have four 1st-level and two 2nd-level spell slots. With an Intelligence of 16, your list of prepared spells can include six spells of 1st or 2nd level, in any combination. If you prepare the 1st-level spell *charm person*, you can cast it using a 1st-level or 2nd-level slot. Casting the spell doesn't remove it from your list of prepared spells.

You can also change your list of prepared spells when you finish a long rest. Preparing a new list of chronomancer spells requires time spent in prayer and meditation: at least 1 minute per spell level for each spell on your list.

\subsubsection{Spellcasting Ability}

Intelligence is your spellcasting ability for your chronomancer spells, since your magic draws upon your knowledge of the timelines and recalling what events you can change and what events you must never alter.
\begin{center}
\textbf{Spell Save DC} = 8 + your Proficiency modifier + your Intelligence modifier\\
\textbf{Spell Attack Bonus} = your Proficiency modifier + your Intelligence modifier
\end{center}

\subsubsection{Ritual Casting}

You can cast a chronomancer spell as a ritual if that spell has the ritual tag and you have the spell prepared.

\subsubsection{Spellcasting Focus}

You can use an astral focus as a spellcasting focus for you chronomancer spells.

\subsection{Unarmored Defense}

Beginning at 1st level, astral energy covers your body creating a layer of protection. While you are not wearing any armor or wielding a shield, your AC equals 10 + your Dexterity modifier + your Intelligence modifier.

\subsection{Timeline}

At 2nd level, you dedicate yourself to a timeline to protect and maintain: The Righteous Timeline, the True Timeline, or the Corrupted Timeline, all detailed at the end of the class description. Your timeline grants you features at 2nd level and again at 6th, 10th, and 14th level.

\subsection{Psychometry}

Starting at 2nd level, you can touch an object or creature and get a temporary glimpse into its past. By focusing on an object or creature for 10 minutes you gain insight into its past. You can ask the DM a number of questions equal to your Intelligence modifier about the past of the object or creature. The DM gives a truthful answer but from the perspective of the object or creature.

Once you use this feature you can't use it again until you finish a long rest.

\subsection{Languages of Time}

Starting at 14th level, you can peer into the timeline and see how languages have come to be allowing you to understand all spoken languages. Moreover, any creature that can understand a language can understand what you say.

\subsection{Planar Walk}

At 18th level, you can pause time for the particles at your feet allowing you to walk on air. You now have a fly speed equal to your movement speed and you can hover in place as long as you are conscious. Additionally, your movement speed is doubled.

\subsection{Timeless Body}

Starting 18th level, your body no longer remains to this time stream, you no longer age and cannot be aged magically. In addition, you no longer need food or water.

\subsection{Astral Infusion}

At 20th level, astral energy flows through your body like blood amplifying your ability to cast spells. If you use your action to cast a spell, you can cast a second spell of 2nd level or lower as a bonus action. When you do so you can't do so again until you finish a short or long rest.

\subsection{Timelines}

A chronomancers job is to maintain the timeline but the ideal timeline is different from chronomancer to chronomancer. There are three main timelines, these are called the Prime Timelines, these timelines tend to reflect the chronomancers personality before they were selected by the twins of Time and Fate to serve them. A chronomancer needs to choose a timeline to follow only upon reaching 2nd level.

\subsection{The Righteous Timeline}

Timekeepers who walk the Righteous Timeline strive to maintain the most positive timeline, they wish to create the best timeline for all living creatures. They focus on manipulating the astral energy to protect themselves and allies. Chronomancers who follow this timeline are usually of the good alignment.

\subsubsection{Defenders Shield}

When you start your path down the Righteous Timeline at 2nd level, you put the lives of your allies before your own. When an ally you can see within 30 feet of you is targeted by an attack you can use your reaction to expend a spell slot of 1st-level or higher in order to increase the target's AC by 5 until the start of your next turn.

You can use this feature a number of times equal to your Intelligence modifier. You regain all expended uses of this feature after you finish a short or long rest.

\subsubsection{Timeline Spells}

Your connection to the astral energy allows you the ability to cast certain spells. At 3rd, 5th, 7th, and 9th level you gain access to timeline spells connected to your timeline choice.

\begin{dndtable}[cX]
\textbf{Timekeeper Level} & \textbf{Timeline Spells} \\
3rd & \textit{enhance ability, protection from poison} \\
5rd & \textit{aura of vitality, remove curse} \\
7rd & \textit{death ward, stoneskin} \\
9rd & \textit{greater restoration, mass cure wounds}
\end{dndtable}

Once you gain access to a timeline spell, you always have it prepared, and it doesn't count against the number of spells you can prepare each day. If you gain access to a spell that doesn't appear on the chronomancer spell list, the spell is nonetheless a chronomancer spell for you.

\subsubsection{The Righteous Cause}

Beginning at 6th level, you can bolster the resolve of your allies. When you do so, choose up to six friendly creatures (which can include yourself) within 30 feet of you who can see or hear you and who can understand you. Each creature gains temporary hit points equal to your level + your Charisma modifier.

Once you use this feature, you can't use it again until you complete a short or long rest.

\subsubsection{Resistance of Time}

Starting at 10th level, you can cause astral energy to surround a creature protecting it against the weathering of time and combat. Choose one damage type when you finish a short or long rest. A creature that you touch gains resistance to that damage type until you choose a different one with this feature. Damage from magical weapons or silvered weapons ignores this resistance.

\subsubsection{Righteous Spell}

Starting at 14th level, your connection to the astral energy seems to strengthen when you cast a spell to aid an ally. Whenever you cast a spell that would heal creatures other than yourself, you can have the creature regain the maximum number of hit points possible from the healing spell. Additionally, each target has advantage on Wisdom saving throws and death saving throws for 1 minute.

You can use this feature a number of times equal to your Intelligence modifier. You regain all expended uses of this feature once you finish a long rest.

\subsection{The True Timeline}

Those who walk the True Timeline know that in order for life to continue advancing there must be balance in life and death, good and evil. The chronomancers who follow this timeline aim to maintain balance, they focus on being able to manipulate the astral energy to damage and harm those who risk disrupting the timeline. These chronomancers are often true neutral alignment, for every wrong there is a right and vice versa.

\subsubsection{Temporal Tremor}

You are spatially attuned to the disruption of time through the True Timeline. Starting at 2nd level, you can focus this disruption on a single target of your choice. You can use your reaction to select one creatue that you can see that is within 30 feet of you, when you do so the target can't take an action this turn. Using this ability, you are considered stunned for your next turn as you must give something away in order to take from another.

Once you use this feature, you can't use it again until you finish a long rest.

\subsubsection{Timeline Spells}

Your connection to the astral energy allows you the ability to cast certain spells. At 3rd, 5th, 7th, and 9th level you gain access to timeline spells connected to your timeline choice.

\begin{dndtable}[cX]
\textbf{Timekeeper Level} & \textbf{Timeline Spells} \\
3rd & \textit{hex, scorching ray} \\
5rd & \textit{hunger of Hadar, Melf's minute meteors} \\
7rd & \textit{banishment, blight} \\
9rd & \textit{flame strike, hold monster}
\end{dndtable}

Once you gain access to a timeline spell, you always have it prepared, and it doesn't count against the number of spells you can prepare each day. If you gain access to a spell that doesn't appear on the chronomancer spell list, the spell is nonetheless a chronomancer spell for you.

\subsubsection{Astral Enhanced Spells}

Beginning at 6th level, when you finish a short or long rest you can selected a damage type: fire, force, lightning, psychic, or necrotic. Whenever you cast a spell of 1st level or higher that deals damage of the type you selected, you add your Intelligence modifier to that damage.

\subsubsection{Astral Form}

Starting at 10th level, you gain the ability to enter a semi-ethereal state while moving.

When you move on your turn, you take only half damage from opportunity attacks, and you can move through any enemy's space but can't willingly end your move there.

On your turn, you can move through any space that is at least 3 inches in diameter and do so without squeezing. When you stop moving, the regular squeezing rules apply if you're in a space one size smaller than you. You can't willingly stop in a space smaller than that, and if you're forced to do so, you immediately flow to the nearest space that can fit you, back along the path of your movement.

\subsubsection{Erase}

At 14th level, you learn how to remove unwanted threads from the True Timeline's web of fate. As an action, you mark a target it must make a Constitution saving throw. On a failed save the target takes 10d10 psychic damage or half as much on a success, as you remove them from the current timeline. If a target is brought to 0 hit points by this feature it dies.

Once you use this feature you can't use it again until you finish a long rest.

\subsection{The Disorderd Timeline}

Chronomancers who tread the Disordered Timeline respect the chaotic nature of the time vortex itself. They gain their power from manipulating the vortex, which is a dangerous thing to meddle with. They manipulate the vortex to allow them to control others and to affect their minds. Chronomancers who follow this timeline are often of chaotic alignment.

\subsubsection{Mental Amendment}

When you start along this timeline at 2nd level, you allow some power from the vortex flow into you which allows you to minutely change the minds of other creatures. When you touch a creature you can release some of your vortex energy in to their mind. For 1 minute, you can either make their perception of you either endearing or nightmarish. If you choose to appear endearing you gain advantage on any Charisma (Persuasion) checks for the duration. If you choose to appear nightmarish, you gain advantage on any Charisma (Intimidation) checks made against the creature for the duration.

Once you use this feature, you can't use it again until you finish a short or long rest.

\subsubsection{Timeline Spells}

Your connection to the astral energy allows you the ability to cast certain spells. At 3rd, 5th, 7th, and 9th level you gain access to timeline spells connected to your timeline choice.

\begin{dndtable}[cX]
\textbf{Timekeeper Level} & \textbf{Timeline Spells} \\
3rd & \textit{crown of madness, suggestion} \\
5rd & \textit{fear, feign death} \\
7rd & \textit{confusion, phantasmal killer} \\
9rd & \textit{dominate person, geas}
\end{dndtable}

Once you gain access to a timeline spell, you always have it prepared, and it doesn't count against the number of spells you can prepare each day. If you gain access to a spell that doesn't appear on the chronomancer spell list, the spell is nonetheless a chronomancer spell for you.

\subsubsection{Glimpse of the Mind}

Starting at 6th level, when you charm a creature you quickly get a glimpse into its mind, learning certain information about its abilities. The DM tells you if the creature is your equal, superior, or inferior in regard to two of the following characteristics of your choice:
\begin{itemize}
\item Intelligence score
\item Wisdom score
\item Charisma score
\item Current hit points
\item Total class levels (if any)
Or you can find out one of the following:
\item Highest ability score
\item Lowest ability score
\item Spellcasting ability (if any)
\end{itemize}

Once you use this feature, you can't use it again until you finish a short or long rest.

\subsubsection{Vortex Soul}

At 10th level, you can release the vortex energy that flows through your body allowing it to affect the minds of creatures around you. As an action, you can infect the minds of creatures around you for 1 minute. When you do so, choose a number of creatures you can see within 30 feet of you, up to a number equal to half your timekeeper level rounded down.

For each creature you have chosen roll a d20, if you roll 11 or more the creature sees you as a loved one and must make a Wisdom saving throw or be charmed by you. A creature charmed by this feature must use its movement to move towards you. If a 10 or lower is rolled, the creature sees you as a horror from its past and must make a Wisdom saving throw or be frightened of you for the duration.

At the end of the targets turn or when the target takes damage, it makes a new Wisdom saving throw against the spell. If the saving throw succeeds, the target is no longer affected and can't be frightened or charmed by you for 24 hours.

Once you use this feature, you must finish a long rest before you can use it again.

\subsubsection{Charmed Focus}

At 14th level, you can use a charmed creature as a focus for your magical power. Before you cast a spell of 2nd level or lower, if there is a creature charmed by you within 10 feet of you that you can see, you can attempt to use its magical energy to support your own. The charmed creature must make a Wisdom or Intelligence saving throw (your choice). A creature with no spellcasting abilities automatically succeeds on this saving throw. On a failed save, the charmed creature loses the spell slot instead of you and the spell is still cast. On a successful save, you must expend the spell slot and the creature is no longer charmed by you.

Once you use this feature, you can't use it again until you must finish a short or long rest.

% ##############################################################
% # ENKINDLER                                                  #
% # source: https://www.dandwiki.com/wiki/Enkindler_(5e_Class) #
% ##############################################################


\twocolumn[{%
\begin{@twocolumnfalse}
\header{The Enkindler Table}
\begin{dndtable}[ l l p{8 cm} l l l l l ]
\textbf{Level} & \textbf{Proficiency} & \textbf{Feature} & \textbf{1st} & \textbf{2nd} & \textbf{3rd} & \textbf{4th} & \textbf{5th} \\
1st & +2 & Spell Casting, Kindle & - & - & - & - & - \\
2st & +2 & Fighting Style & 2 & - & - & - & - \\
3st & +2 & Focus & 3 & - & - & - & - \\
4st & +2 & Ability Score Improvement & 3 & - & - & - & - \\
5st & +3 & Extra Attack, Fiery Verdict & 4 & 2 & - & - & - \\
6st & +3 & Burning Soul & 4 & 2 & - & - & - \\
7st & +3 & Focus Feature & 4 & 3 & - & - & - \\
8st & +3 & Ability Score Improvement & 4 & 3 & - & - & - \\
9st & +4 & Aesthetic & 4 & 3 & 2 & - & - \\
10st & +4 & Flames From Within & 4 & 3 & 2 & - & - \\
11st & +4 & Focus Feature & 4 & 3 & 3 & - & - \\
12st & +4 & Ability Score Improvement & 4 & 3 & 3 & - & - \\
13st & +5 & - & 4 & 3 & 3 & 1 & - \\
14st & +5 & Body of Ash & 4 & 3 & 3 & 1 & - \\
15st & +5 & Focus Feature & 4 & 3 & 3 & 2 & - \\
16st & +5 & Ability Score Improvement & 4 & 3 & 3 & 2 & - \\
17st & +6 & - & 4 & 3 & 3 & 3 & 1 \\
18st & +6 & Supreme Flames & 4 & 3 & 3 & 3 & 1\\
19st & +6 & Ability Score Improvement & 4 & 3 & 3 & 3 & 2 \\
20st & +6 & Ashen Soul & 4 & 3 & 3 & 3 & 2
\end{dndtable} 
\end{@twocolumnfalse}
}]

\section{Enkindler}

\subsection{Creating an Enkindler}

Enkindlers are Mages that focus only on the element of Fire and use Fire Magic, it can be anything having to do with fire from just starting a spark easily to causing a scale wide Inferno and the like, over time, by studying fire books and fire magics and learning the art of fire, one can become a Enkindler, however, that usually means enduring hot conditions and weathers, and usually be in hot places, often where there is Fire, and usually enduring Temperature ranges in Boiling point.

Enkindlers Spells can burn plants, set places on fire, burn skin, turn things to ash and cause burns, and if their Fire is strong enough, melt Metals, and if their Fire magics are strong enough and imbue with rocks and stone, can turn it into Lava or Molten Rocks. Their area of Expertise is always Fire Magics, and all of a Enkindler's Spells must be Fire Magic. Powerful rechargeable Fire "grenade" and decent melee make the Enkindler a good choice for offensive players.

\newpage

\subsubsection{Quick Build}

\textbf{TODO}

\subsection{Class Features}

\paragraph{HIT POINTS}\mbox{}\\
\textbf{Hit Dice:} 1d6 per Enkindler level\\
\textbf{Hit Points at 1st Level:} 6 + your Constitution modifier\\
\textbf{Hit Points at Higher Levels:} 1d6 (or 4) + your  Constitution modifer per Enkindler level after 1st

\paragraph{PROFICIENCIES}\mbox{}\\
\textbf{Armor:} All armor, shields\\
\textbf{Weapons:} Simple weapons, martial weapons\\
\textbf{Tools:} None\\
\textbf{Saving Throws:} Constitution, Wisdom\\
\textbf{Skills:} Choose any three from Acrobatics, Arcana, Athletics, Deception, Insight, Intimidation, Investigation, Perception, and Survival

\subsubsection{Equipmet}
You start with the following equipment, in addition to the equipment granted by your background:
\begin{itemize}
\item (a) Chain Shirt or (b) Chain Mail
\item (a) Longsword or (b) Rapier
\item (a) Heavy Crossbow + 20 bolts or (b) Pike 
\item (a) Shield or (b) Arcane Focus
\end{itemize}

\subsection{Spellcasting}

Your spellcasting as an Enkindler focuses on imbuing your constructs with magical powers upon crafting them and using that power within to unleash the true potential of your magic. You choose your spells from the Enkindler spells list.

\subsubsection{Preparing and Casting Spells}

The Enkindler table shows how many spell slots you have to cast your spells of 1st level and higher. To cast one of these spells, you must expend a slot of the spell’s level or higher. You regain all expended spell slots when you finish a long rest.

At level 1 the Enkindler knows two Cantrips: Create Bonfire, Control Flames, Greenflame Blade, Fire Bolt, Produce Flame.\\
At level 2 the Enkindler knows two level 1 spells: Burning Hands, Faerie Fire, Fog Cloud, Hellish Rebuke, Longstrider, Shield.\\
At level 5 the Enkindler knows three level 2 spells: Aganazzar's Scorcher, Continual Flame, Darkvision, Flame Blade, Flaming Sphere, Heat Metal, Pyrotechnics, Scorching Ray.\\
At level 9 the Enkindler knows three level 3 spells: Elemental Weapon (fire only), Fear, Fireball, Flame Arrow, Gaseous Form, Haste, Melf's Minute Meteors, Stinging Cloud.\\
At level 13 the Enkindler knows two level 4 spells: Arcane Eye, Conjure Minor Elemental (fire only), Fire Shield, Vitriolic Sphere Wall of Fire.\\
At level 17 the Enkindler knows one level 5 spells: Cloud Kill, Conjure Elemental (fire only), Flame Strike, Immolation, Passwall, Scrying.\\

\subsubsection{Spellcasting Ability}
Wisdom is your spellcasting ability for your Enkindler spells, since your spells come from your inner fortitude and strength. You use your Wisdom whenever a spell refers to your spellcasting ability. In addition, you use your Wisdom modifier when setting the saving throw DC for a Enkindler spell you cast and when making an attack roll with one.

\begin{center}
\textbf{Spell save DC} = 8 + your proficiency bonus + your Wisdom modifier\\
\textbf{Spell attack modifier} = your proficiency bonus + your Wisdom modifier
\end{center}

\subsection{Kindle}

Starting at level 1 you gain the ability to kindle your self at every level including level one. For every point of kindle you have you increase fire damage by 1 point. Every time you kindle you lose maximum hit points equal to the amount of kindling you have done. You can choose to kindle up to your Enkindler level. Once you choose to kindle you can not undo your decision these hit point losses are permanent and so is the damage boost.

\subsection{Fighting Style}

Starting at level 2 you adopt a particular style of fighting as your specialty. Choose one of the following options. You can’t take a Fighting Style option more than once, even if you later get to choose again.

\subsubsection{Archery}

You gain a +2 bonus to attack rolls you make with ranged weapons.

\subsubsection{Defense}

While you are wearing armor, you gain a +1 bonus to AC.

\subsubsection{Dueling}

When you are wielding a melee weapon in one hand and no other weapons, you gain a +2 bonus to damage rolls with that weapon.

\subsubsection{Great Weapon Fighting}

When you roll a 1 or 2 on a damage die for an attack you make with a melee weapon that you are wielding with two hands, you can reroll the die and must use the new roll, even if the new roll is a 1 or a 2. The weapon must have the two-handed or versatile property for you to gain this benefit.

\subsubsection{Two-Weapon Fighting}

When you engage in two-weapon fighting, you can add your ability modifier to the damage of the second attack.

\subsection{Focus}
Starting at level 3 you get to choose your Focus. You gain a feature from your focus again on 7th, 11th and 15th level.

\subsection{Ability Score Increase}

When you reach 4th level, and again at 8th, 12th, 16th and 19th level, you can increase one ability score of you choice by 2, or you can increase two ability scores of your choice by 1. As normal, you can't increase an ability score above 20 using this feature.

\subsection{Extra Attack}

Starting at level 5 you can attack twice, instead of once, whenever you take the attack action on your turn.

\subsection{Fiery Verdict}

Starting at level 5 you may tap directly into your own lifeforce for power, fueling your flames with your vitality. When casting a spell, instead of expending a spell slot you may instead lose hit points equal to the Spell Level + 1. This loss cannot be reduced in anyway.

\subsection{Burning Soul}

Starting at level 6 you gain resistance to fire damage. At level 12 this turns into fire immunity.

\subsection{Aesthetic}

Starting at level 9 you may choose for any Fire damage dealt by your character to be Radiant damage instead.

\subsection{Flames From Within}

Starting at level 10 your body has become the only thing keeping the torrent of fire that is your soul from destroying everything around it. When you die, the effects of the fireball spell occur centered on your space, as though you had cast the spell using your your highest remaining spell slot. You may make a Wisdom saving throw against your spell save DC to stop this from happening.

\subsection{Body of Ash}

Your internal flames become as a star, providing all the energy your body needs. Starting at level 14 you no longer need to drink water, eat food or sleep.

\subsection{Supreme Flame}

Starting at level 18 you can reroll damage on a fire-based spell when you cast it. You must take the second result.

\subsection{Ashen Soul}

Starting at level 20 you deal 1d6 fire damage to any creature that deals damage to you. You still take damage as normal.

\subsection{Focus}

\subsection{Ebony}

\subsubsection{Keeper of the Flame}

Starting at level 3 you gain one extra cantrip and one extra level one spell.

\subsubsection{Lingering Flame}

Starting at level 7 you can choose to cast any level one spell or higher as a mine. A spell cast as a mine will stay in the position you left it in for one hour. The mine will wait for an aggressive enemy to move within 15 feet, when the enemy gets close enough the mine will cast the spell from it's location at the target. You expel your spell slot at the creation of the mine.

\subsubsection{Flame of the King}

Starting at level 11 you can take 4 points of damage to cast the fire bolt spell as a bonus action once per round.

\subsubsection{Acute Bomb}

Starting at level 15 you can take damage equal to half your total health. If this damage drops you below zero you can activate the Flames From Within ability as a bonus action. You then return to one hit point after making one successful death saving throw.

\subsection{Iron}

\subsubsection{Combustion}

Starting at level 3 you can use your action to burn all creatures within 5 feet of you equal to your kindle bonus plus your spellcasting bonus. Creatures affected must make a dex save against your spellcasting DC.

\subsubsection{War Casting}

Starting at level 7 when you use your action to cast a cantrip, you can make one weapon attack as a bonus action.

\subsubsection{Controlled Flames}

Starting at level 15 you gain when you use your action to cast a spell, you can make one weapon attack as a bonus action.

\subsection{Ivory}

\subsubsection{Carthus Flame Arc}

Starting at level 3 you add half your kindle bonus rounded down to weapon attacks.

\subsubsection{Life in the Flame}

Starting at level 7 you gain 20 extra hit points.

\subsubsection{Improved Criticals}

Starting at level 11 your weapon attacks score a critical hit on a roll of 19 or 20.

\subsubsection{Chaos Storm}

Starting at level 15 You become deeply attuned to your fire magics. When your health drops below 0, you can activate Flames From Within and double your kindle damage to the blast. This ability can only be utilized once per long rest.

\subsection{Multiclassing}
\textbf{Prerequisites} To qualify for multiclassing into the Enkindler class, you must meet these prerequisites: 13 wisdom\\
\textbf{Proficiencies}  When you multiclass into the class, you gain the following proficiencies: All Armors\\